\chapter{Resumen.}
\justifying%
En los últimos años, el desarrollo de entornos virtuales complejos—como \textit{REBOUND}, \textit{Universe sandbox}, \textit{Stellarium} y \textit{Celestia}—ha experimentado un crecimiento exponencial en sus capacidades. Sin embargo, al abordar la estabilidad y escalabilidad en sistemas celestes de \textit{n} cuerpos, por ejemplo, la simulación de nuevas formas de formación de galaxias, persisten importantes desafíos. Los modelos actuales que simulan estos fenómenos normalmente no permiten modificar parámetros clave de los cuerpos celestes—como velocidad, posición y masa—durante la ejecución. Esta limitación restringe la capacidad de generar escenarios físicamente más precisos, especialmente en sistemas atípicos de $n$ cuerpos.

En este proyecto terminal, proponemos una solución a esta restricción mediante la construcción de un modelo que posibilita la modificación en tiempo real de la masa en un sistema de dos cuerpos. El modelo incorpora diversas técnicas de cálculo gravitatorio, incluyendo el cálculo directo y el algoritmo de Barnes-Hut, así como algoritmos bioinspirados. La solución propuesta podría emplearse para simular interacciones físicas dinámicas en entornos virtuales.

\chapter{Abstract.}
In recent years, the development of complex virtual environments, such as \textit{REBOUND}, \textit{Universe sandbox}, \textit{Stellarium} and \textit{Celestia}, has experienced exponential growth in capabilities. However, when addressing stability and scalability in $n$-body celestial systems, such as the simulation of novel forms of galaxy formation, significant challenges remain. Current models that simulate these phenomena typically do not allow changes to key parameters of celestial bodies, such as velocity, position, and mass, during runtime. This limitation restricts the ability to generate more physically accurate scenarios, especially in atypical $n$-body systems.

In this terminal project, we propose a solution to this limitation by constructing a model that enables modification of mass in a two-body system during execution time. The model incorporates various gravitational computation techniques, including direct calculation and the Barnes-Hut algorithm, as well as bio-inspired algorithms. The proposed solution could be used to simulate dynamic physical interactions in virtual environments.
