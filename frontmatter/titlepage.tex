\begin{titlepage}
    \begin{center}
        % Logo and Institute Name placement
        \raisebox{0.0\height}{
        \begin{adjustbox}{max width=1.2\linewidth, right}
          \begin{minipage}[c]{0.5\linewidth}
            \centering
           \raisebox{0.0\height}{\includegraphics[width=0.5\linewidth]{img/logo-ipn.png}}
          \end{minipage}\hfill
          \begin{minipage}[c]{1.5\linewidth}
            \centering
            {\fontsize{28}{30}\selectfont\textsc{Instituto Politécnico Nacional}}\\[2mm]
            {\fontsize{24}{26}\selectfont\textsc{Escuela Superior de Cómputo}}\\[2mm]
            {\fontsize{20}{22}\selectfont\textsc{Subdirección Académica}}
        \end{minipage}
        \hfill
          \begin{minipage}[c]{0.5\linewidth}
            \centering
            \raisebox{0.0\height}{\includegraphics[width=0.8\linewidth]{img/EscudoESCOM.png}}
          \end{minipage}
        \end{adjustbox}}

        \begin{adjustwidth}{-2.5cm}{}
            \begin{center}
                \begin{tabular}{c@{\hspace{8cm}}c}
                    No.\ de TT:\ 2025-B065 &
                    \today
                \end{tabular}\\
                \vspace{0.55cm}
                \textbf{Documento técnico} \\
                
                \vspace{0.5cm}
                
                % Title
                \begin{minipage}{0.8\textwidth}
                    \centering
                    {\fontsize{16}{16}\selectfont\textbf{``Modelo para representar comportamientos gravitacionales con dos cuerpos''\\}}
                \end{minipage}
                
                \vspace{0.3cm}
                
                % Presentan
                {\fontsize{16}{16}\selectfont\textit{Presentan:}} \\
                \vspace{0.25cm}
                % Authors (names in alphabetical order by last name)
                {\fontsize{14}{14}\selectfont\textbf{
                José Emiliano Carrillo Barreiro.\footnote{\href{mailto:carrillobarjosee@gmail.com}{carrillobarjosee@gmail.com}}\\
                José Ángel Robles Otero.\footnote{\href{mailto:robles.otero.jose.angel@gmail.com}{robles.otero.jose.angel@gmail.com}}}
                }
                \vspace{0.5cm}
                
                % Directors
                {\fontsize{14}{14}\selectfont\textsc{Directores:\\}}
                
                \vspace{10pt}
                
                % Director names
                {\fontsize{14}{14}\selectfont\textbf{Dr.\ Cesar Hernández Vasquez\\}}
                
                \vspace{5pt}
                
                % Co-director (if applicable)
                {\fontsize{14}{14}\selectfont\textbf{Dr.\ Mauricio Olguín Carbajal}}
                
                \vspace{0.5cm}
                
                % Abstract
                \begin{minipage}{\textwidth}
                    \begin{center}
                        \textbf{Resumen:}\\[0.3cm]
                    \end{center}
                    {\fontsize{12}{12}\selectfont
                    En los últimos años, la elaboración de entornos virtuales complejos, como \textit{REBOUND}, \textit{Universe sandbox}, \textit{Stellarium} y \textit{Celestia}, han experimentado un crecimiento exponencial en sus capacidades. Más, cuando se habla de implementar mejoras en la precisión y escalabilidad dentro de sistemas de \textit{n}-cuerpos celestes, por ejemplo, la simulación de nacimientos de galaxias, todavía existen dificultades, ya que, actualmente, los modelos que los simulan no permiten introducir cambios en los parámetros de los cuerpos durante su ejecución, parámetros como la velocidad, posición y masa del cuerpo, lo que impide contar con escenarios más fidedignos a los comportamientos  físicos dentro de sistemas de $n$-cuerpos atípicos. En este proyecto, se propone solucionar este problema mediante la construcción de un modelo que permita realizar cambios al parámetro más significativo del  sistema, compuesto por dos cuerpos, en tiempo de ejecución, implementando técnicas como el método multipolar rápido (FMM, por sus siglas en inglés, \textit{Fast Multipole Method}), el \textit{algoritmo de Barnes-Hut} y algoritmos bioinspirados. La solución que se desarrolle podría usarse para simular comportamientos físicos dentro de entornos virtuales.\\

                     \textbf{Palabras clave: } Modelado y Simulación de sistemas, problema de los $n$-cuerpos, Mecánica Celeste, y Algoritmos Bioinspirados.}
                \end{minipage}
                
                \vspace{0.7cm}
                
                % Date
                \textbf{\large Fecha: \today}
            \end{center}
        \end{adjustwidth}
    \end{center}
\end{titlepage}