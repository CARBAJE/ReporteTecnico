\begin{adjustbox}{width=\textwidth,center}
\begin{tikzpicture}[
    node distance=1.2cm and 0.8cm,
    state/.style={rectangle, draw, rounded corners, minimum width=2.5cm, minimum height=1cm, align=center},
    choice/.style={diamond, draw, aspect=2, minimum width=1.5cm, minimum height=0.75cm, inner sep=0pt},
    carril/.style={rectangle, draw=gray!50, thick, minimum width=3.5cm, minimum height=26cm, fill=gray!10},
    carrillabel/.style={above, font=\bfseries\small, anchor=south},
    arrow/.style={-stealth, thick, >=latex},
    initial/.style={circle, fill=black, inner sep=2pt},
    final/.style={double, circle, draw, inner sep=1.5pt}
]

% Definir carriles
\node[carril] (usuario) at (-7, 0) {};
\node[carrillabel] at (usuario.north) {Usuario};

\node[carril] (interfaz) at (-3.5, 0) {};
\node[carrillabel] at (interfaz.north) {
    \begin{tabular}{c}
        Interfaz de\\
        Usuario
    \end{tabular}
};

\node[carril] (bio) at (0, 0) {};
\node[carrillabel] at (bio.north) {
	\begin{tabular}{c}
		Algoritmo \\
		Bioinspirado
	\end{tabular}
};

\node[carril] (rebound) at (3.5, 0) {};
\node[carrillabel] at (rebound.north) {REBOUND};

\node[carril] (visual) at (7, 0) {};
\node[carrillabel] at (visual.north) {
	\begin{tabular}{c}
		Módulo de \\
		Visualización
	\end{tabular}
};

% Estados Usuario
\node[state] (IniciarPrograma) at (-7, 11) {Iniciar Programa};
\node[state, below=0.5cm of IniciarPrograma] (DefinirParametros) {Definir Parámetros};

% Estados Interfaz
\node[state] (CapturaParametros) at (-3.5, 8.5) {Captura Parámetros};
\node[state, below=20.3 cm of CapturaParametros] (MostrarResultados) {Mostrar Resultados};

% Estados Bioinspirado
\node[state] (InicializarPoblacion) at (0, 7.5) {Inicializar Población};
\node[state, below=0.5cm of InicializarPoblacion] (GenerarNuevaGeneracion) {Generar Nueva \\Generación};
\node[state, below=15cm of GenerarNuevaGeneracion] (SeleccionarMejores) {Seleccionar Mejores};
\node[choice, below=0.5cm of SeleccionarMejores] (CriterioParada) {Criterio Parada};

% Estados REBOUND
\node[state] (CrearNuevaSimulacion) at (3.5, 6) {Crear Nueva \\Simulación};
\node[state, below=0.5cm of CrearNuevaSimulacion] (AgregarCuerpos) {Agregar Cuerpos};
\node[state, below=0.5cm of AgregarCuerpos] (DefinirCondiciones) {Definir Condiciones};
\node[state, below=0.5cm of DefinirCondiciones] (IniciarSimulacion) {Iniciar Simulación};
\node[state, below=0.5cm of IniciarSimulacion] (ResolverEcuaciones) {Resolver Ecuaciones};
\node[state, below=0.5cm of ResolverEcuaciones] (ActualizarEstado) {Actualizar Estado};
\node[choice, below=0.5cm of ActualizarEstado] (VerificarTiempo) {
    \begin{tabular}{c}
        Verificar Tiempo\\
        Visualización
    \end{tabular}
};
\node[choice, below=0.5cm of VerificarTiempo] (ContinuarSimulacion) {
    \begin{tabular}{c}
        Continuar\\
        Simulación
    \end{tabular}
};
\node[state, below=0.5cm of ContinuarSimulacion] (EvaluarFitness) {Evaluar Fitness};

% Estados Visualización
\node[state] (RecolectarDatos) at (7, -5) {Recolectar \\Datos};
\node[state, below=0.5cm of RecolectarDatos] (ProcesarDatos) {Procesar \\Datos};
\node[state, below=0.5cm of ProcesarDatos] (GenerarGraficos) {Generar \\Gráficos};

% Puntos inicial y final destacados
\node[initial] (start) at ($(IniciarPrograma.north)+(0,1)$) {};
\node[final] (end) at ($(MostrarResultados.south)+(0,-.5)$) {};

% Transiciones principales
\draw[arrow] (start) -- (IniciarPrograma);
\draw[arrow] (IniciarPrograma) -- (DefinirParametros);
\draw[arrow] (DefinirParametros) -| (CapturaParametros);
\draw[arrow] (CapturaParametros) -| (InicializarPoblacion);
\draw[arrow] (InicializarPoblacion) -- (GenerarNuevaGeneracion);
\draw[arrow] (GenerarNuevaGeneracion) -- (CrearNuevaSimulacion);
\draw[arrow] (CrearNuevaSimulacion) -- (AgregarCuerpos);
\draw[arrow] (AgregarCuerpos) -- (DefinirCondiciones);
\draw[arrow] (DefinirCondiciones) -- (IniciarSimulacion);
\draw[arrow] (IniciarSimulacion) -- (ResolverEcuaciones);
\draw[arrow] (ResolverEcuaciones) -- (ActualizarEstado);
\draw[arrow] (ActualizarEstado) -- (VerificarTiempo);

% Transiciones de decisión mejoradas
\draw[arrow] (VerificarTiempo.east) -| node[right] {Sí} (RecolectarDatos);
\draw[arrow] (VerificarTiempo.west) -- node[right] {No} 
    ($(VerificarTiempo.west)+(0,-1.5)$) |- (ContinuarSimulacion.north);

\draw[arrow] (RecolectarDatos) -- (ProcesarDatos);
\draw[arrow] (ProcesarDatos) -- (GenerarGraficos);
\draw[arrow] (GenerarGraficos.south) -- ++(0,-0.5) -| ($(ContinuarSimulacion.east)+(0.25,0)$) |- (ContinuarSimulacion.east);(ContinuarSimulacion.east);

% Conexiones mejoradas para evitar solapamientos
\draw[arrow] (ContinuarSimulacion.west) -- node[above] {Sí} 
    ($(ContinuarSimulacion.west)-(1,0)$) |- (ResolverEcuaciones.west);
    
\draw[arrow] (ContinuarSimulacion) -- node[right] {No} (EvaluarFitness);

% Transiciones finales ajustadas
\draw[arrow] (EvaluarFitness) -| (SeleccionarMejores);
\draw[arrow] (SeleccionarMejores) -- (CriterioParada);
\draw[arrow] (CriterioParada.west) -- 
    node[above, pos=0.3] {No} 
    ($(CriterioParada.west)-(1.2,0)$) coordinate (a) 
    |- ($(GenerarNuevaGeneracion.south) + (-1.2,-7.5)$) coordinate (b)
    -| (GenerarNuevaGeneracion.south);
\draw[arrow] (CriterioParada.south) -- node[below] {Sí} (MostrarResultados.east);
\draw[arrow] (MostrarResultados) -- (end);

\end{tikzpicture}
\end{adjustbox}

