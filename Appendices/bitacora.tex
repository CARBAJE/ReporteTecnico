\section{Pruebas realizadas}
En esta sección se detallan las diversas pruebas llevadas a cabo durante el desarrollo de este proyecto. Para cada prueba, se describe el objetivo que se persigue, la implementación realizada, los resultados esperados y los resultados obtenidos.

\subsection{Primeros pasos con REBOUND}
La primera prueba consistió en una ejecución básica de la biblioteca REBOUND. El objetivo principal fue familiarizarnos con su funcionamiento y explorar los diferentes integradores disponibles. Esto nos permitiría evaluar cuál se adapta mejor a las necesidades específicas de nuestro proyecto.

Para esto, se buscó simular la órbita de dos cuerpos con masas iguales, definiendo como condiciones iniciales a la velocidad y la posición y finalmente se utiliza matplot para tener una representacion grafica de las trayectorias.
\subsubsection{Resultados}
Tras realizar este primer acercamiento a REBOUND, observamos que el integrador IAS15 presentaba la mejor relación entre precisión y velocidad en nuestras pruebas iniciales. Esta observación nos convenció de que sería el integrador más adecuado para utilizar en el desarrollo de nuestro proyecto. Adicionalmente, este primer contacto nos permitió familiarizarnos con la herramienta, facilitando su posterior uso a lo largo del proyecto.

\begin{listing}[H]
\begin{minted}[linenos, breaklines, breakanywhere]{python}
import rebound
import numpy as np
import matplotlib.pyplot as plt

# Crear la simulación y establecer la constante gravitacional (G = 1 en unidades arbitrarias)
sim = rebound.Simulation()
sim.G = 1.0

# Definir las masas (masas similares)
m1 = 1.0
m2 = 1.0

# Para que el período del movimiento relativo sea de 1 año:
# T = 2*pi*sqrt(a_rel^3/(m1+m2)) = 1 => a_rel^3 = 1/(2*pi^2)
a_rel = (1.0 / (2 * np.pi**2))**(1/3)

# Cada cuerpo orbita a una distancia r del centro de masas:
r = a_rel / 2.0

# Velocidad orbital para órbita circular: v = sqrt(G/(4r)) (para cuerpos de igual masa)
v = np.sqrt(1.0 / (4 * r))

# Agregar las partículas con condiciones iniciales:
# Se colocan simétricamente respecto al origen.
# Cuerpo 1: posición (-r, 0) y velocidad perpendicular (0, -v)
# Cuerpo 2: posición (r, 0) y velocidad (0, v)
sim.add(m=m1, x=-r, y=0, z=0, vx=0, vy=-v, vz=0)
sim.add(m=m2, x=r, y=0, z=0, vx=0, vy=v, vz=0)

# Tras agregar las partículas, mover el sistema al centro de masas (opcional pero recomendable)
sim.move_to_com()

# Configurar el integrador (se usa "ias15" por su alta precisión)
sim.integrator = "ias15"

# Podemos definir un dt (aunque con "ias15" no es estrictamente necesario)
sim.dt = 0.001

# Definir el tiempo final de la simulación: 1 año (equivalente a 365 días en este sistema de unidades)
t_final = 1.0

# Crear un arreglo de tiempos para almacenar las posiciones (por ejemplo, 1000 puntos a lo largo del año)
n_outputs = 1000
times = np.linspace(0, t_final, n_outputs)

# Listas para almacenar las posiciones de cada cuerpo
x1, y1 = [], []
x2, y2 = [], []

# Integrar la simulación y almacenar las posiciones
for t in times:
    sim.integrate(t)
    x1.append(sim.particles[0].x)
    y1.append(sim.particles[0].y)
    x2.append(sim.particles[1].x)
    y2.append(sim.particles[1].y)

# Graficar las trayectorias
plt.figure(figsize=(8,8))
plt.plot(x1, y1, label="Cuerpo 1")
plt.plot(x2, y2, label="Cuerpo 2")
plt.xlabel("x")
plt.ylabel("y")
plt.title("Simulación de dos cuerpos en 1 año")
plt.legend()
plt.axis("equal")
plt.grid(True)
plt.show()
\end{minted}
\caption{Ejemplo de código Rebound para simulación de dos cuerpos}
\label{lst:rebound_example}
\end{listing}

\subsection{Calculo de métricas}
Como segunda prueba, nos enfocamos en el cálculo de distintas métricas que nos podrian ayudar a decidir cuando un sistema ya es establ, ademas de esto, para esta segunda prueba en lugar de dar todos las condiciones iniciales, solo se utiliza la masa inicial y las trayectorias que sigue cada cuerpo.

\subsubsection{Resultados}
Calculamos el semieje mayor, la excentricidad, la energía relativa y el exponente de Lyapunov y tras ver como cambiaban estos datos con respecto al tiempo, decidimos que la métrica con la que vamos a trabajar será el exponente de Lyapunov, siendo que cuando este exponente sea menor a 0.1 se considerara que el sistema se encuentra en equilibrio.

\begin{listing}[H]
\begin{minted}[linenos, breaklines, breakanywhere]{python}
if __name__ == "__main__":
    # Parámetros de simulación
    t_max = 1e3  # Años
    N_steps = 1000
    inc_deg = 30  # Inclinación en grados para la órbita del planeta

    # Ejecutar simulación
    times, a_arr, e_arr, energy, a_pert, e_pert, x, y, z = run_simulation(t_max, N_steps, inc_deg)

    # Calcular métricas
    metrics = calculate_stability(times, a_arr, e_arr, energy, a_pert, e_pert)

    # Generar gráficos de análisis
    plot_analysis(times, a_arr, e_arr, energy, a_pert, e_pert, metrics)

    # Generar animación 3D (en ventana separada)
    animate_trajectory(x, y, z, times, save_path="trayectoria.gif", save_format="gif")
\end{minted}
\caption{Ejemplo de código Rebound con cálculo de métricas}
\label{lst:rebound_metrics}
\end{listing}


\subsection{Primeras pruebas con algoritmos bioinspirados}
Ya que nos familiarizamos con REBOUND y analizamos las métricas calculadas, empezamos a bocetar como seria el uso de los algoritmos bioinspirados dentro de nuestro proyecto, por lo que realizamos una prueba rapida con el algoritmo NSGA2 como primera opcion, ya que consideramos el problema como una optimizacion multiobjetivo.

\subsubsection{resultados}
Nos dimos cuenta de que con el uso de tecnologias como pymoo, la tarea de realizar el algoritmo de optimizacion se facilitaba bastante, ya que contabamos con una estructura bastante solida y solo nos tendriamos que preocupar por las adecuaciones especificas para nuestro proyecto, ademas de que se obtuvo una ejecucion bastante rapida, siendo de 0.3 segundos para optimizar con el cálculo de 100 generaciones de 50 individuos.

\begin{listing}[H]
\begin{minted}[linenos, breaklines, breakanywhere]{python}
if __name__ == "__main__":
    # Crear la configuración de la simulación usando variables globales.
    config = SimulationConfig()
    sim_runner = SimulationRunner(config)

    # Lista de funciones objetivo (buscando estabilidad, sin target específico)
    objectives = [
        objective_variation_a,
        objective_lyapunov  # Esta función requiere el runner y opt_vars
    ]

    # Usar la lista global de restricciones
    constraints = GLOBAL_CONSTRAINTS

    # Definir el problema de optimización.
    problem = ModularOptimizationProblem(
        sim_runner,
        n_free_bodies=3,  # 3 cuerpos no centrales en GLOBAL_BODIES_CONFIG
        objective_funcs=objectives,
        constraint_funcs=constraints
    )

    # Seleccionar NSGA-II como algoritmo multiobjetivo
    algorithm = NSGA2(pop_size=50)

    # Ejecutar la optimización (ajustar el criterio de terminación según sea necesario)
    res = minimize(problem, algorithm, termination=('n_gen', 100), seed=1, verbose=True)

    # Mostrar resultados
    print("Soluciones no dominadas (condiciones iniciales):")
    print(res.X)
    print("\nValores de los objetivos (variación en a y λ):")
    print(res.F)
    print("\nRestricciones (deben ser <= 0):")
    print(res.G)
\end{minted}
\caption{Sección principal del script de optimización multi-objetivo}
\label{lst:main_optimization}
\end{listing}
