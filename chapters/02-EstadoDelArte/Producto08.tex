\section[Producto 08: Algoritmo TPM para simulaciones N-cuerpos]{Algoritmo Tree Particle-Mesh (TPM) para simulaciones N-cuerpos cosmológicas}

El algoritmo \textit{Tree Particle-Mesh} (TPM) combina dos enfoques clásicos en dinámica gravitacional: el método de árbol (\textit{Tree}) para el cálculo directo de fuerzas en agrupaciones jerárquicas de partículas, y el método de partículas en malla (\textit{Particle-Mesh}, PM), altamente eficiente para computar fuerzas gravitacionales en grandes volúmenes espaciales. Esta integración permite un balance efectivo entre precisión y eficiencia computacional, siendo especialmente útil en simulaciones cosmológicas con millones o miles de millones de partículas.

\subsection{Descripción y Arquitectura Técnica}

El algoritmo TPM aprovecha la linealidad de las fuerzas gravitacionales respecto al campo de densidad para aplicar una descomposición de dominio basada en la distribución de densidad. De esta manera, el campo se divide en múltiples regiones de alta densidad espacialmente localizadas, que contienen una fracción significativa de la masa total pero ocupan un volumen reducido.

En estas regiones densas, la fuerza gravitacional se calcula mediante el algoritmo de árbol, complementado por fuerzas de marea provenientes del resto del dominio. Para el resto del volumen, donde la densidad es baja, se emplea el algoritmo PM con pasos de tiempo más amplios, optimizando así el uso de recursos.

El uso de diferentes escalas de tiempo —pasos grandes para el componente PM y pasos adaptativos más cortos para las regiones árbol— permite una integración eficiente y precisa de la dinámica de partículas. Además, debido a la independencia de las regiones árbol, el algoritmo puede ser fácilmente paralelizado usando técnicas de paso de mensajes (\textit{message passing}), haciéndolo ideal para entornos de cómputo distribuido.

\subsection{Aplicaciones Prácticas}

El algoritmo TPM ha sido ampliamente utilizado en simulaciones cosmológicas centradas en la formación de estructuras a gran escala, como cúmulos de galaxias, halos de materia oscura y la red cósmica de filamentos y vacíos. Su diseño lo hace especialmente útil para estudios que requieren alta resolución en regiones localizadas de alta densidad, sin comprometer la eficiencia global del cómputo.

Además, el enfoque del TPM también puede extenderse a otros dominios fuera de la cosmología, siempre que la distribución de densidad permita segmentar el dominio en grupos relativamente aislados de partículas. Esto lo hace aplicable en problemas de dinámica de sistemas granulares, plasma, o simulaciones de fluidos donde existan regiones con alta concentración de materia rodeadas de volúmenes menos densos.

\subsection{Ventajas y Desventajas}

Entre las principales ventajas del algoritmo TPM se encuentra su capacidad de balancear de manera eficiente la resolución espacial con el uso de recursos computacionales. La segmentación del espacio según la densidad permite asignar mayor detalle a regiones físicamente relevantes, mientras que el uso del método PM en zonas menos densas conserva la eficiencia. Asimismo, la posibilidad de paralelizar de forma natural las regiones árbol mediante paso de mensajes lo convierte en una herramienta escalable para supercomputación.

Sin embargo, también existen algunas desventajas. La implementación del TPM puede ser más compleja que la de otros algoritmos más homogéneos, como P3M o solo PM, debido al manejo de múltiples escalas temporales y dominios dinámicamente segmentados. Además, en situaciones donde la densidad no presenta fuertes contrastes o donde las regiones densas no están bien aisladas, el beneficio del algoritmo puede verse reducido, y otros métodos podrían ser más apropiados.

\subsection{Relevancia para el Proyecto de Simulación Gravitacional}

La estrategia híbrida del algoritmo TPM es altamente relevante para proyectos que requieren combinar escalabilidad con fidelidad física. Su capacidad de segmentar dinámicamente el dominio espacial según la densidad permite aplicar distintos niveles de resolución según la necesidad, lo cual es ideal para simulaciones adaptativas.

Además, su compatibilidad con computación paralela y su menor tiempo de ejecución respecto a algoritmos tradicionales como P3M lo hacen especialmente útil para integrar con métodos de optimización, análisis masivo de datos simulados o incluso para alimentar sistemas de aprendizaje automático que requieran datos de alta resolución espacial en regiones seleccionadas del espacio simulado.
