\section[Métodos Hidrodinámicos Sin Malla]{Implementación de Métodos Hidrodinámicos Sin Malla en PKDGRAV3 para Simulaciones Cosmológicas}%
\label{sec:state_of_the_art_04}

Desde la perspectiva de las simulaciones de sistemas dinámicos, como los comportamientos gravitacionales de cuerpos celestes descritos en este proyecto, la eficiencia computacional y la precisión física son pilares fundamentales. El proyecto aquí presentado busca modelar interacciones gravitacionales entre dos cuerpos con la capacidad de modificar dinámicamente parámetros como masa, posición y velocidad durante la ejecución, un desafío que requiere métodos innovadores para optimizar recursos sin comprometer la fidelidad física. En el marco de lo anterior, el artículo ``Mesh-free hydrodynamics methods for astrophysical simulations: the PKDGRAV3 code'' de I. Alonso Asensio et al~\cite{AlonsoAsensio2022} presenta la implementación de métodos hidrodinámicos sin malla (\textit{mesh-free}) en PKDGRAV3, un enfoque inicialmente diseñado para simulaciones cosmológicas que ofrece perspectivas valiosas y adaptables al modelado de sistemas gravitacionales.

\subsection{Descripción y Arquitectura Técnica}

El método presentado en el artículo surge como una alternativa a las simulaciones tradicionales que dependen de mallas estructuradas, las cuales limitan la representación de interacciones complejas en sistemas dinámicos debido a su alta demanda computacional. En lugar de calcular las fuerzas entre todos los elementos del sistema en cada iteración mediante una malla global, los métodos sin malla, como \textit{Meshless Finite Mass} (MFM) y \textit{Meshless Finite Volume* (MFV)}, emplean partículas para discretizar el fluido, resolviendo las ecuaciones hidrodinámicas con un enfoque adaptativo basado en los vecinos más cercanos. Este diseño permite capturar fenómenos complejos, como choques y discontinuidades, sin necesidad de una malla fija, reduciendo significativamente la complejidad computacional. Además, la optimización del código mediante algoritmos de búsqueda de vecinos y paralelización avanzada ajusta el modelo para mantener la precisión física con un manejo eficiente de recursos.

\subsection{Características y Aplicaciones Prácticas}

El método sin malla en PKDGRAV3 se caracteriza por su capacidad para simular sistemas complejos, como la formación de estructuras cosmológicas, con una precisión notable incluso al manejar grandes contrastes de densidad. Su independencia respecto a una malla fija permite incorporar fenómenos dinámicos sin un incremento exponencial en el tiempo de cálculo, una ventaja clave para sistemas variables. En el contexto de este proyecto, que busca simular interacciones gravitacionales con parámetros modificables, el enfoque sin malla podría inspirar estrategias para optimizar el cálculo de fuerzas gravitacionales, especialmente en escenarios donde la masa o la posición de los cuerpos cambian en tiempo real. Aunque su aplicación original se centra en hidrodinámica, la lógica subyacente es trasladable a sistemas gravitacionales, donde la eficiencia y la adaptabilidad son igualmente críticas.

\subsection{Ventajas y Desventajas}

Entre las ventajas del método sin malla destacan su escalabilidad a sistemas complejos, la preservación de la precisión física con una resolución adaptativa y la posibilidad de manejar simulaciones a gran escala mediante optimizaciones paralelas. Estas características lo convierten en una herramienta prometedora para simulaciones que requieren flexibilidad computacional. Sin embargo, presenta desventajas notables: su validación se ha restringido a sistemas cosmológicos específicos, lo que plantea dudas sobre su generalización a configuraciones más intrincadas. Adicionalmente, la precisión depende de una selección cuidadosa de parámetros computacionales, lo que podría complicar su implementación en sistemas gravitacionales altamente dinámicos sin un ajuste riguroso.

\subsection{Relevancia para el Proyecto de Simulación Gravitacional}

El proyecto descrito en este reporte técnico tiene como objetivo desarrollar un modelo que permita ajustar dinámicamente parámetros de dos cuerpos celestes, utilizando técnicas como el Método Multipolar Rápido (FMM) y el algoritmo de Barnes-Hut. Aunque los métodos sin malla se diseñaron para simulaciones hidrodinámicas, su enfoque en reducir la complejidad computacional sin sacrificar precisión física resulta altamente relevante. La capacidad de ajustar el modelo mediante una discretización flexible podría adaptarse para optimizar el cálculo de interacciones gravitacionales, complementando las técnicas propuestas en el proyecto. Por ejemplo, combinar MFM y MFV con FMM y Barnes-Hut podría mejorar la eficiencia al manejar cambios en la masa o posición de los cuerpos, permitiendo simulaciones más rápidas y adaptativas.

\subsection{Experiencia Personal y Opinión}

Al evaluar conceptualmente el método sin malla en el marco de este proyecto, no se realizó una implementación directa, pero se analizaron sus principios aplicados a simulaciones gravitacionales. Inferimos que el enfoque de Alonso Asensio et al.\ representa un avance significativo en la optimización de simulaciones de sistemas de múltiples cuerpos. Su énfasis en la eficiencia y la flexibilidad sugiere un gran potencial para nuestro modelo, especialmente en la gestión de parámetros dinámicos. La idea de reducir la dependencia de estructuras fijas sin perder precisión física es particularmente atractiva para simulaciones en tiempo real. No obstante, la necesidad de calibrar cuidadosamente los parámetros del método podría ser un obstáculo en sistemas gravitacionales con comportamientos altamente variables, aunque las optimizaciones presentadas ofrecen una vía prometedora para superar esta limitación.