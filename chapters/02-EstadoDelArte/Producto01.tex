\textit{Framework} \texttt{ode\_num\_int}]{Producto 01: \textit{Framework} \texttt{ode\_num\_int} para Integración Numérica}\label{sec:ode_num_int}

En el ámbito de la simulación de sistemas gravitacionales, la integración numérica de ecuaciones diferenciales ordinarias (EDOs) desempeña un papel fundamental al describir el movimiento de cuerpos bajo influencias gravitacionales. La precisión y eficiencia de los métodos de integración utilizados impactan directamente la capacidad de los modelos para representar fenómenos físicos complejos, como los comportamientos dinámicos de sistemas de dos cuerpos. En este contexto, el \textit{framework} \texttt{ode\_num\_int}, presentado en el artículo ``C++ Playground for Numerical Integration Method Developers"~\cite{Orlov2017}, emerge como una herramienta relevante para el desarrollo y evaluación de métodos de integración numérica personalizados, con potencial aplicabilidad al proyecto descrito en este documento.

\subsection{Descripción y Arquitectura Técnica}

El \textit{framework} \texttt{ode\_num\_int} es una biblioteca de software desarrollada en \texttt{C++11}, diseñada para proporcionar a investigadores y desarrolladores un entorno flexible y extensible para la creación y prueba de métodos de integración numérica destinados a resolver EDOs. A diferencia de bibliotecas tradicionales como GSL o Boost.Odeint, que priorizan soluciones numéricas predefinidas, \texttt{ode\_num\_int} se centra en el proceso investigativo, ofreciendo una arquitectura modular basada en clases \textit{template}. Esta estructura permite a los usuarios personalizar componentes individuales según las necesidades específicas de sus simulaciones.

La arquitectura del \textit{framework} se organiza en tres pilares principales:
\begin{enumerate} 
    \item Una infraestructura común que incluye el patrón observador para la gestión dinámica de \textit{callbacks}, \textit{holders} de propiedades para el manejo flexible de variables, factorías para la creación de instancias, parámetros opcionales y utilidades de temporización. 
    \item Componentes de álgebra lineal con \textit{templates} para vectores y matrices dispersas, soporte para factorización $LU$ y almacenamiento optimizado de datos. 
    \item Capacidades de resolución numérica que abarcan \textit{solvers} para sistemas algebraicos no lineales, métodos de iteración tipo Newton, \textit{solvers} explícitos e implícitos para EDOs, manejo de eventos y controladores de tamaño de paso. 
\end{enumerate}
Esta modularidad permite combinar diferentes elementos de manera sencilla, facilitando la experimentación con métodos diversos.

\subsection{Características y Aplicaciones Prácticas}

Entre las características distintivas de \texttt{ode\_num\_int} se encuentra su flexibilidad para integrar componentes de \textit{solver}, su monitoreo de rendimiento integrado y su soporte para múltiples tipos de métodos numéricos, incluyendo explícitos, implícitos y basados en extrapolación. Estas capacidades lo hacen adecuado para simulaciones de sistemas físicos complejos. Un ejemplo concreto de su aplicación es la simulación de la dinámica de transmisiones continuamente variables (CVT), un sistema con 3,600 variables de estado que involucra modelos de cuerpos elásticos deformables e interacciones de fricción. En este caso, se observó que el método del trapecio con un tamaño de paso de \(10^{-5}\) alcanzó una precisión comparable a la del método Runge-Kutta de cuarto orden (RK4) con un tamaño de paso de \(10^{-8}\), lo que indica un potencial de mejora en la eficiencia computacional.

El \textit{framework} se utiliza principalmente en campos como la computación científica, simulaciones de ingeniería, modelado de sistemas mecánicos y análisis de dinámica en aeronáutica y robótica. Su diseño lo posiciona como una herramienta valiosa para investigadores que requieren un alto grado de control sobre los métodos de integración.

\subsection{Ventajas y Desventajas}

El \texttt{ode\_num\_int} ofrece varias ventajas significativas. Su alta extensibilidad permite a los usuarios incorporar nuevos componentes y métodos, mientras que su enfoque en el rendimiento, respaldado por herramientas de monitoreo, facilita la optimización de simulaciones. Además, su carácter \textit{open-source} bajo la licencia GNU GPL fomenta la colaboración y el desarrollo comunitario. Sin embargo, presenta limitaciones notables: está diseñado principalmente para sistemas de escala media, lo que podría restringir su uso en simulaciones de gran escala como las de \(n\)-cuerpos masivos, y requiere un conocimiento avanzado de \texttt{C++}, lo que puede limitar su accesibilidad. Actualmente, se enfoca en métodos de un solo paso, aunque se planea ampliar esta capacidad en futuras versiones.

\subsection{Relevancia para el Proyecto de Simulación Gravitacional}

En el contexto del trabajo terminal abordado en este reporte, el cual busca desarrollar un modelo para simular comportamientos gravitacionales de dos cuerpos con modificación dinámica de parámetros de posición y velocidad inicial, \texttt{ode\_num\_int} podría desempeñar un papel complementario. La simulación de sistemas gravitacionales requiere integrar las ecuaciones de movimiento, que son EDOs, y combinarlas con técnicas como el Método Multipolo Rápido (FMM) y el algoritmo de Barnes-Hut para calcular fuerzas gravitacionales de manera eficiente. La flexibilidad de \texttt{ode\_num\_int} para adaptar métodos de integración y su capacidad de monitoreo de rendimiento podrían facilitar la implementación de integradores personalizados que soporten cambios dinámicos en los parámetros durante la ejecución, un objetivo central del proyecto.

%\subsection{Experiencia Personal y Opinión}

%Para evaluar el framework, se realizó una prueba exploratoria basada en su documentación y código fuente disponible en GitHub. Se configuró un entorno en \texttt{C++} para simular un sistema simple de dos cuerpos con parámetros iniciales fijos, utilizando un \textit{solver} explícito proporcionado por el framework. La experiencia demostró que la modularidad del \texttt{ode\_num\_int} permite una rápida adaptación de los métodos de integración, aunque la configuración inicial requiere tiempo debido a la complejidad de las clases *template*. En nuestra opinión, este \textit{framework} es una herramienta poderosa para investigadores con experiencia en \texttt{C++}, ofreciendo un control excepcional sobre el proceso de integración. Sin embargo, su curva de aprendizaje podría ser un obstáculo para usuarios menos experimentados, y su enfoque en sistemas de escala media sugiere que podría requerir modificaciones para aplicaciones más ambiciosas dentro del proyecto.
