
\section[Modelo de Estabilidad Planetaria]{Modelo de Simulación de Estabilidad Planetaria Basado en Uniformidad de Masas}%
\label{sec:state_of_the_art_10}

Dentro de los trabajos recientes que abordan la estabilidad dinámica en sistemas planetarios y su relación con las características intrínsecas del sistema, destaca el enfoque presentado por Dong-Hong Wu, Sheng Jin y Jason H. Steffen en su artículo ``Enhanced Stability in Planetary Systems with Similar Masses'', publicado en \textit{The Astronomical Journal}~\cite{Wu2025}. Este trabajo se centra en investigar cuantitativamente la conexión entre la estabilidad dinámica a largo plazo de sistemas multiplanetarios y la uniformidad de las masas de los planetas que los componen. Su propósito principal es explorar, mediante simulaciones numéricas de N-cuerpos, si los sistemas planetarios con masas más similares (mayor uniformidad) exhiben intrínsecamente una mayor estabilidad, ofreciendo así una posible explicación basada en sesgos de supervivencia para la tendencia observada de ``guisantes en una vaina'' (\textit{peas-in-a-pod}) en los sistemas detectados por la misión Kepler.

\subsection{Características Principales}
El modelo computacional descrito en~\cite{Wu2025} se basa fundamentalmente en el código N-cuerpos \textit{REBOUND}~\cite{Rein2012}, utilizando específicamente el integrador híbrido \textit{Mercurius}~\cite{rein2019}, adecuado para simulaciones de larga duración que involucran posibles encuentros cercanos. Las simulaciones parten de sistemas idealizados compuestos por ocho planetas orbitando una estrella de masa solar. Las condiciones iniciales establecen órbitas circulares y coplanares, una configuración común en muchos sistemas Kepler. La masa total de los planetas se mantiene constante en 30 masas terrestres ($M_\oplus$) en la mayoría de las simulaciones, explorando también casos con 10 $M_\oplus$ y 100 $M_\oplus$.

Un parámetro clave es la separación mutua entre planetas adyacentes, $K$, medida en unidades del radio de Hill mutuo (Ecuaciones 1 y 2 en~\cite{Wu2025}), que se mantiene constante dentro de cada sistema simulado y se varía entre 3 y 10 en el conjunto de experimentos. Para cuantificar la uniformidad de masas planetarias, los autores emplean el índice de Gini ajustado ($\mathcal{G}_m$, Ecuación 3 en~\cite{Wu2025}), donde $\mathcal{G}_m = 0$ representa masas idénticas y $\mathcal{G}_m$ cercano a 1 indica que una sola masa domina el sistema. Se generan conjuntos de sistemas con valores de $\mathcal{G}_m$ distribuidos uniformemente entre 0 y aproximadamente 0.97 para cada valor de $K$ estudiado.

Las simulaciones se integran hasta que ocurre un ``encuentro cercano'', definido como una separación entre dos planetas menor que el radio de Hill del planeta más interno, o hasta alcanzar un tiempo máximo de integración de $10^8$ años. El tiempo hasta este evento se registra como el ``tiempo de inestabilidad dinámica'' ($t$), que luego se escala por el período orbital del planeta más interno ($t_0$). El análisis principal se enfoca en la relación entre $t/t_0$ y $\mathcal{G}_m$ para diferentes valores de $K$, comparando sistemas con masas desiguales frente a sistemas de control con masas iguales. Investigaciones adicionales consideran sistemas ``bien ordenados'', donde las masas planetarias aumentan monotónicamente hacia el exterior.\ \textbf{Es fundamental señalar que, según la descripción metodológica en~\cite{Wu2025}, el modelo no contempla la modificación de parámetros de los cuerpos (masa, posición, velocidad) durante el tiempo de ejecución de la simulación; la evolución se sigue pasivamente desde las condiciones iniciales hasta el punto de inestabilidad.}

\subsection[Ventajas]{Ventajas}
El estudio~\cite{Wu2025} establece una correlación estadísticamente significativa entre la uniformidad de masas y la estabilidad del sistema para configuraciones no resonantes (específicamente para $K > 4$), proporcionando una fuerte evidencia cuantitativa para la hipótesis de que los sistemas más uniformes son inherentemente más estables. Esta es una contribución importante para explicar las arquitecturas observadas en exoplanetas. El uso del índice de Gini ajustado ($\mathcal{G}_m$) como métrica de uniformidad es una elección metodológica clara y efectiva para comparar diversas distribuciones de masa. La implementación sobre \textit{REBOUND} con el integrador \textit{Mercurius} sugiere una base computacionalmente eficiente, capaz de manejar las escalas de tiempo prolongadas requeridas para estos estudios de estabilidad dinámica. Además, el trabajo aborda explícitamente el papel de las resonancias de movimiento medio (MMRs), identificando cómo estas pueden alterar la tendencia general de estabilidad, como se visualiza en la Figura 1 de~\cite{Wu2025}. Los autores también proponen un mecanismo físico plausible para la menor estabilidad de sistemas no uniformes: la equipartición de la energía aleatoria, que tiende a excitar las excentricidades de los planetas de menor masa, conduciendo a encuentros cercanos más tempranos.

\subsection[Desventajas y Limitaciones]{Desventajas y Limitaciones}
Una limitación explícita reconocida por los autores~\cite{Wu2025} es que la simulación se detiene en el \textit{primer} encuentro cercano detectado. Esto impide el estudio de la evolución dinámica posterior, que podría incluir colisiones planetarias, eyecciones, o incluso la reconfiguración hacia un estado estable diferente. Este criterio de parada simplifica la dinámica post-inestabilidad, que puede ser crucial en la formación final de la arquitectura del sistema.

\textbf{La limitación más relevante desde la perspectiva de nuestro proyecto principal}, inferida directamente de la metodología descrita en~\cite{Wu2025}, es la \textbf{ausencia total de capacidad para modificar los parámetros de los cuerpos celestes (masa, velocidad, posición) durante la ejecución de la simulación}. El modelo opera bajo un paradigma de ``condiciones iniciales fijas'', evolucionando el sistema pasivamente. Esto lo hace fundamentalmente inadecuado para escenarios que requieran simular eventos dinámicos interactivos o cambios paramétricos en tiempo real, como la acreción de masa, efectos de fuerzas no conservativas variables, o la introducción/eliminación de cuerpos, que son objetivos centrales de nuestro trabajo.

Otras limitaciones incluyen una menor sensibilidad del tiempo de inestabilidad al índice $\mathcal{G}_m$ para valores bajos de $K$ ($K \leq 4$), donde la alta densidad de MMRs domina la dinámica (como se observa en la Figura 1 y 2 de~\cite{Wu2025}). Además, aunque se exploran sistemas ``bien ordenados'', el enfoque principal en distribuciones aleatorias o perfectamente iguales podría no capturar toda la diversidad de configuraciones iniciales posibles post-formación. Finalmente, los propios autores señalan en su discusión~\cite{Wu2025} que, si bien sus resultados coinciden parcialmente con las observaciones de pares resonantes ($G_m < 0.38$), no pueden explicar completamente por qué estos pares observados tienden a ser \textit{más} uniformes que los pares no resonantes, sugiriendo que la dinámica simulada (posiblemente debido a la detención temprana o la falta de modelado de colisiones) podría estar incompleta.

\subsection[Ámbito de Uso]{Ámbito de Uso}
El modelo y las simulaciones presentadas en~\cite{Wu2025} están diseñados específicamente para la investigación teórica de la estabilidad dinámica a largo plazo de sistemas planetarios multi-cuerpo, con un enfoque particular en arquitecturas compactas y coplanares similares a las descubiertas por Kepler. Su utilidad reside en estudios estadísticos para entender cómo las propiedades intrínsecas del sistema (separación, uniformidad de masa) influyen en las tasas de supervivencia y, por ende, en las características observables de la población de exoplanetas. Es una herramienta para probar hipótesis sobre la evolución dinámica pasiva y los sesgos de observación.

\subsection[Resultados Clave]{Resultados Clave}
Los hallazgos cuantitativos principales incluyen:
\begin{itemize}
    \item Para sistemas non-resonantes y $K > 4$, existe una fuerte anticorrelación entre el logaritmo del tiempo de inestabilidad ($\log(t/t_0)$) y el índice de Gini ($\mathcal{G}_m$). Sistemas con mayor uniformidad (menor $\mathcal{G}_m$) son significativamente más estables. Por ejemplo, para K=6.5, el coeficiente de correlación de Spearman es de -0.69 con $p < 10^{-4}$ (Figura 2 en~\cite{Wu2025}).
    \item Los sistemas con masas desiguales son generalmente menos estables que sus contrapartes de masas iguales para el mismo $K$ (siempre que no estén dominados por MMRs específicas). La diferencia en tiempo de estabilidad puede ser de varios órdenes de magnitud (Figura 1 en~\cite{Wu2025}).
    \item Cerca de MMRs de primer orden (ej., 4:3, 5:4), la relación entre estabilidad y $\mathcal{G}_m$ es más compleja. En algunos casos (como cerca de 5:4, K=7.5), la anticorrelación casi desaparece, mientras que en otros (como cerca de 4:3, K=9.73) persiste pero es más débil (Figura 3 en~\cite{Wu2025}).
    \item Para sistemas ``bien ordenados'' cerca de la MMR 5:4 (K $\approx$ 7.5), la máxima estabilidad se alcanza para un rango intermedio de $\mathcal{G}_m$ ($0.2 < G_m < 0.4$), lo cual difiere del comportamiento en sistemas con masas distribuidas aleatoriamente (Figura 5 en~\cite{Wu2025}). Este resultado se alinea mejor con las observaciones de pares resonantes ($G_m < 0.38$).
\end{itemize}

\subsection{Relevancia para el Proyecto de Simulación Gravitacional}
El trabajo de Wu \textit{et al.}~\cite{Wu2025} es relevante para nuestro proyecto en varios aspectos contextuales, aunque sus objetivos y metodología difieren fundamentalmente. Proporciona un ejemplo contemporáneo de cómo se utilizan herramientas avanzadas de simulación N-cuerpos (\textit{REBOUND}, \textit{Mercurius}) y métricas específicas ($\mathcal{G}_m$) para abordar cuestiones astrofísicas complejas, como la estabilidad dinámica de sistemas exoplanetarios. El estudio demuestra la importancia de las condiciones iniciales y las propiedades intrínsecas del sistema (como la distribución de masas) en la determinación de su evolución a largo plazo.

Sin embargo, la \textbf{relevancia directa} para la implementación técnica de nuestro proyecto es limitada debido a una diferencia conceptual clave: el modelo de Wu \textit{et al.}~\cite{Wu2025} opera bajo la premisa de parámetros de cuerpo (masa, posición, velocidad) \textbf{fijos durante la ejecución}, excepto por su evolución natural bajo la gravedad mutua. Su objetivo es estudiar la estabilidad \textit{inherente} de configuraciones iniciales dadas, no simular sistemas donde los parámetros cambian debido a procesos físicos adicionales (como acreción, colisiones con modificación de masa) o intervenciones externas simuladas.

Nuestro proyecto, en cambio, se enfoca precisamente en superar esta limitación, buscando desarrollar un modelo que \textbf{permita la modificación de parámetros clave, especialmente la masa, en tiempo de ejecución}. El análisis del trabajo de Wu~\cite{Wu2025} subraya la necesidad de arquitecturas de simulación diferentes cuando el objetivo es modelar sistemas N-cuerpos ``atípicos'' o procesos dinámicos que involucran cambios paramétricos activos. Mientras que~\cite{Wu2025} ilustra el estado del arte en simulaciones de estabilidad pasiva a largo plazo, nuestro proyecto se orienta hacia la flexibilidad y la capacidad de interacción dinámica durante la simulación, un área donde el modelo analizado no ofrece una solución aplicable. Por lo tanto, sirve como un punto de referencia importante sobre las capacidades (y limitaciones) de los enfoques estándar para un tipo diferente de problema de simulación N-cuerpos.
