\section[Método n-NNN]{Método de Red de n-Vecinos Más Cercanos (n-NNN) para Simulaciones Moleculares}%
\label{sec:state_of_the_art_03}

En la diciplina de las simulaciones de sistemas dinámicos, como los comportamientos gravitacionales de cuerpos celestes descritos en este proyecto, la eficiencia computacional y la precisión física son pilares fundamentales. El proyecto aquí presentado busca modelar interacciones gravitacionales entre dos cuerpos con la capacidad de modificar dinámicamente parámetros como masa, posición y velocidad durante la ejecución, un desafío que requiere métodos innovadores para optimizar recursos sin comprometer la fidelidad física. En este contexto, el artículo ``Artificial Intelligent Molecular Dynamics and Hamiltonian Surgery''~\cite{Maguire2005} introduce el método de Red de n-Vecinos Más Cercanos (n-NNN), un enfoque inicialmente diseñado para simulaciones moleculares que ofrece perspectivas valiosas y adaptables al modelado de sistemas gravitacionales.

\subsection{Descripción y Arquitectura Técnica}

El método n-NNN surge como una alternativa a las simulaciones tradicionales que dependen de Hamiltonianos aditivos por pares, los cuales limitan la representación de interacciones complejas de n-cuerpos debido a su alta demanda computacional. En lugar de calcular las fuerzas entre todos los elementos del sistema en cada iteración, n-NNN emplea matrices multidimensionales para almacenar las fuerzas de cada sitio en función de su vecindario local, asemejándose a una red neuronal. Este diseño permite capturar distribuciones espaciales de múltiples cuerpos sin necesidad de incluir el Hamiltoniano completo, reduciendo significativamente la complejidad computacional. Además, la ``cirugía Hamiltoniana'' propuesta por los autores facilita la selección de términos de orden superior relevantes, ajustando el modelo para mantener la precisión estructural con un número reducido de vecinos considerados.

\subsection{Características y Aplicaciones Prácticas}

El método n-NNN se caracteriza por su capacidad para simular sistemas complejos, como líquidos de Lennard-Jones\footnote{Los líquidos de Lennard-Jones son sistemas modelados en los que las interacciones entre las partículas (átomos o moléculas neutras) se describen mediante el potencial de Lennard-Jones. Para una definición a profundidad, cosulte:~\cite{Allen2017}} o sistemas iónicos\footnote{Los sistemas iónicos son aquellos en los que las interacciones predominantes se deben a la atracción electrostática entre iones de carga opuesta. En aras de una definición comprehensiva, consúltese:~\cite{Atkins2008}}, con una precisión notable incluso al limitar el número de vecinos analizados. Su independencia respecto al grado de no aditividad permite incorporar términos de múltiples cuerpos sin un incremento exponencial en el tiempo de cálculo, una ventaja clave para sistemas dinámicos. En el contexto de este proyecto, que busca simular interacciones gravitacionales con parámetros modificables, el enfoque n-NNN podría inspirar estrategias para optimizar el cálculo de fuerzas gravitacionales, especialmente en escenarios donde la masa o la posición de los cuerpos cambian en tiempo real. Aunque su aplicación original se centra en dinámicas moleculares, la lógica subyacente es trasladable a sistemas gravitacionales, donde la eficiencia y la adaptabilidad son igualmente críticas.

\subsection{Ventajas y Desventajas}

Entre las ventajas del método n-NNN destacan su escalabilidad a sistemas complejos, la preservación de la precisión estructural con un número reducido de vecinos y la posibilidad de ``entrenar'' redes aplicables a diversos estados del sistema. Estas características lo convierten en una herramienta prometedora para simulaciones que requieren flexibilidad computacional. Empero, presenta desventajas notables: su validación se ha restringido a sistemas relativamente simples, como líquidos de Lennard-Jones y cloruro de sodio, lo que plantea dudas sobre su generalización a configuraciones más intrincadas. Por otra parte, la precisión depende de una selección cuidadosa del número de vecinos y de la función generadora, lo que podría complicar su implementación en sistemas gravitacionales altamente dinámicos sin un ajuste riguroso.

\subsection{Relevancia para el Proyecto de Simulación Gravitacional}

El trabajo terminal descrito en este reporte técnico tiene como objetivo desarrollar un modelo que permita ajustar dinámicamente parámetros de dos cuerpos celestes, utilizando técnicas como el Método Multipolar Rápido (FMM) y el algoritmo de Barnes-Hut. Aunque el n-NNN se diseñó para simulaciones moleculares, su enfoque en reducir la complejidad computacional sin sacrificar precisión física resulta altamente relevante. La capacidad de ajustar el modelo mediante ``cirugía Hamiltoniana'' podría adaptarse para optimizar el cálculo de interacciones gravitacionales, complementando las técnicas propuestas en el proyecto. Por ejemplo, combinar n-NNN con FMM y Barnes-Hut podría mejorar la eficiencia al manejar cambios en la masa o posición de los cuerpos, permitiendo simulaciones más rápidas y adaptativas.

\subsection{Experiencia Personal y Opinión}

Al evaluar conceptualmente el método n-NNN en el marco de este proyecto, no se planea realizar una implementación directa, pero se analizaron sus principios aplicados a simulaciones gravitacionales. Consideramos que el enfoque de Maguire y Woodcock representa un avance significativo en la optimización de simulaciones de sistemas de múltiples cuerpos. Su énfasis en la eficiencia y la flexibilidad sugiere un gran potencial para nuestro modelo, especialmente en la gestión de parámetros dinámicos. La idea de reducir el número de interacciones consideradas sin perder precisión estructural es particularmente atractiva para simulaciones en tiempo real. No obstante, la necesidad de calibrar cuidadosamente los parámetros del método podría ser un obstáculo en sistemas gravitacionales con comportamientos altamente variables, aunque la ``cirugía Hamiltoniana'' ofrece una vía prometedora para superar esta limitación.