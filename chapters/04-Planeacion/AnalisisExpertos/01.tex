\subsection[Consulta con experto núm. 01]{Consulta con Experto en Optimización Evolutiva e Investigador de agentes en Videojuegos}%
\label{subsec:validacion-experto}

Con el objetivo de validar la pertinencia del problema abordado y la adecuación de la solución propuesta en este Trabajo Terminal, se realizó una consulta especializada. Esta validación externa es crucial para asegurar que el proyecto se alinea con las necesidades y desafíos actuales en los campos de la simulación y sus potenciales aplicaciones.

\subsubsection{Perfil del Experto Consultado}

Se consultó al Maestro en Ciencias Computacionales (M.CC.) José Alberto Torres León, egresado del Centro de Investigación en Cómputo (CIC) del Instituto Politécnico Nacional, y actualmente estudiante de último año del Doctorado en Ciencias de la Computación en la misma institución. Sus áreas de especialización incluyen la generación procedimental de contenido, agentes inteligentes para videojuegos (jugadores y dinámicos), aprendizaje por refuerzo, optimización evolutiva y \textit{Deep Learning}. Su experiencia en optimización y su profundo conocimiento del desarrollo y las limitaciones en el ámbito de los videojuegos lo convierten en un validador idóneo para el presente proyecto.

\subsubsection{Método de Consulta}

La consulta se llevó a cabo mediante una reunión formal el martes 13 de mayo de 2025, entre las 12:35 y las 13:10 horas, en las instalaciones de investigación del CIC.\ Previo a la discusión, el M.CC.\ Torres León revisó el borrador del presente reporte técnico. Durante la reunión, se le expuso verbalmente el propósito, la visión y la evolución del proyecto, desde su concepción inicial como un ``videojuego que permitiera jugar con las cualidades físicas gravitacionales de objetos para atraer, repeler, o que se mantengan en un equilibrio modificando características de los objetos en tiempo de ejecución'', hasta su iteración final como un ``modelo de simulación de dos cuerpos celestes que permita el cambio de características en los cuerpos para encontrar los valores idóneos que mantengan al sistema en estabilidad''.

\subsubsection{Resumen de Opiniones y Observaciones Obtenidas}

Tras la revisión del material y la discusión, el M.CC.\ Torres León aportó las siguientes observaciones y valoraciones:

\begin{itemize}
    \item \textbf{Comprensión del Problema y Enfoque:}
    \begin{itemize}
        \item Inicialmente, el experto señaló que la afirmación de que el estado del arte actual en simuladores no permite la definición propia de parámetros no era del todo precisa. Se clarificó que la limitación principal que el proyecto busca abordar es la \textit{modificación de estos parámetros en tiempo de ejecución}.
        \item Identificó correctamente que el problema subyacente que se trata de resolver es de naturaleza combinatoria y subrayó la necesidad de una definición formal y clara de dicho problema combinatorio dentro del reporte.
        \item Enfatizó la importancia de establecer un objetivo específico para la evaluación de resultados, que permita corroborar que el comportamiento simulado sea el adecuado y esperado.
    \end{itemize}

    \item \textbf{Aplicabilidad y Relevancia en Videojuegos:}
    \begin{itemize}
        \item Confirmó que las herramientas actuales para motores físicos en el desarrollo de videojuegos ``no poseen flexibilidad en el entorno de desarrollo'' para el tipo de interacciones dinámicas que este proyecto plantea, limitando la implementación de mecánicas como las concebidas originalmente para el Trabajo Terminal.
        \item Señaló que la industria de los videojuegos busca realizar simulaciones apegadas a comportamientos físicos reales, pero explorando ``casos y resultados que no se han captado o no pasan en la realidad'', lo cual se alinea con la capacidad del modelo propuesto.
        \item Consideró plausible que modelos similares ya existan en videojuegos comerciales, pero como parte de ``patentes o secretos industriales'', citando como ejemplo el jefe final del videojuego \textit{Bayonetta 1}, donde cuerpos celestes son dirigidos dinámicamente. Esto resalta la dificultad de acceso a este conocimiento y la relevancia de una propuesta abierta.
        \item Opinó que la solución propuesta puede ``dar una ilusión más realista de comportamientos gravitacionales en videojuegos'', con un ``mucho más valor en juegos de tipo plataformero''.
        \item Prevé una ``influencia positiva en la experiencia del usuario y del desarrollador'' al contar con modelos de esta naturaleza.
        \item Destacó un ``sesgo en la investigación'' del área debido a que muchos avances no se documentan públicamente o están protegidos, limitando el acceso a información pública. Mencionó que la investigación en generación procedimental de contenido y videojuegos ha ganado seriedad recientemente (desde aproximadamente 2016), aunque persiste cierto escepticismo en algunos círculos académicos.
    \end{itemize}

    \item \textbf{Planeación y Delimitación del Proyecto:}
    \begin{itemize}
        \item El experto consideró que el proyecto presenta una ``buena dirección'' y una ``buena delimitación'', especialmente tras las sucesivas adaptaciones del alcance.
    \end{itemize}

    \item \textbf{Notas Adicionales:}
    \begin{itemize}
        \item Mencionó la existencia de optimizadores basados en fenómenos gravitacionales, aunque no especificó nombres.
        \item Observó que el enfoque del reporte técnico parecía más orientado hacia la Ingeniería en Sistemas Computacionales (ISC) que hacia la Ingeniería en Inteligencia Artificial (IIA), una apreciación relevante para la presentación del trabajo.
    \end{itemize}
\end{itemize}

\subsubsection{Análisis de la Validación Experta y Vinculación con el Proyecto}

Las observaciones del M.CC.\ José Alberto Torres León proporcionan una validación significativa tanto para la relevancia del problema como para la viabilidad y el impacto potencial de la solución propuesta.

\begin{itemize}
    \item \textbf{Validación de la Pertinencia del Problema:}
    \begin{itemize}
        \item La confirmación de que las herramientas actuales en motores de videojuegos carecen de la flexibilidad necesaria para modificar parámetros físicos gravitacionales en tiempo de ejecución (como se propone) valida directamente la necesidad que este proyecto busca satisfacer.
        \item Su comentario sobre la existencia de soluciones propietarias o no publicadas refuerza la justificación de desarrollar un modelo abierto y documentado que pueda servir como base para futuras investigaciones o desarrollos, especialmente considerando el sesgo de investigación existente.
        \item La identificación del problema como combinatorio y la necesidad de su definición clara son aportes que refinan la presentación del desafío técnico que se aborda.
    \end{itemize}

    \item \textbf{Respaldo a la Solución Propuesta:}
    \begin{itemize}
        \item La opinión de que el modelo puede ofrecer una ``ilusión más realista'' y tener un ``mayor valor'' en ciertos géneros de videojuegos respalda el potencial de aplicación de la solución.
        \item La valoración positiva sobre la ``buena dirección'' y ``delimitación'' del proyecto, tras conocer su evolución, sugiere que el enfoque actual es pertinente y manejable.
        \item La necesidad de un objetivo específico de evaluación de resultados, sugerida por el experto, se alinea con la metodología de investigación y se considerará fundamental para la etapa de pruebas y validación del modelo.
    \end{itemize}
\end{itemize}

Gracias a la consulta con el M.CC.\ Torres León validamos que el problema de la modificación dinámica de parámetros gravitacionales en simulaciones es relevante, especialmente con miras a aplicaciones innovadoras como en videojuegos, y que la solución propuesta, aunque delimitada a dos cuerpos, aborda una brecha existente y tiene una dirección metodológica sólida. Las sugerencias sobre la definición del problema combinatorio y los criterios de evaluación serán integradas para fortalecer el rigor del proyecto.