\section{Análisis de Requerimientos}

La sección de requerimientos constituye el pilar fundamental para el desarrollo exitoso de cualquier modelo o software, ya que establece de manera clara y estructurada las expectativas sobre lo que el modelo debe hacer y cómo debe comportarse. En este sentido, los requerimientos se clasifican en dos grandes categorías: \textbf{funcionales} y \textbf{no funcionales}. Los requerimientos funcionales (RF) detallan las acciones específicas que el modelo debe ejecutar, como la modificación dinámica de parámetros o la simulación de interacciones gravitacionales. Por su parte, los requerimientos no funcionales (RNF) especifican las cualidades y limitaciones del modelo, incluyendo aspectos como el rendimiento, la escalabilidad, la usabilidad y la compatibilidad con hardware determinado.

En este reporte técnico, se presenta un análisis detallado del contenido textual para identificar tanto los requerimientos explícitos como los implícitos, garantizando que el modelo de simulación gravitacional cumpla con los objetivos establecidos. Entre estos objetivos se incluye la capacidad de ajustar dinámicamente la masa de los cuerpos celestes y asegurar el cumplimiento de las leyes de la física y la mecánica celeste. Además, se han tomado en cuenta factores esenciales como la eficiencia computacional, la estabilidad de las simulaciones y la accesibilidad para usuarios sin experiencia técnica avanzada.

La definición precisa y la documentación adecuada de estos requerimientos no solo orientarán el proceso de desarrollo, sino que también facilitarán la validación y verificación del modelo una vez implementado. Esto asegura que el producto final esté alineado con las necesidades y expectativas de los usuarios, ya sean investigadores, académicos o profesionales del sector.

\newpage
\subsection{Requerimientos Funcionales (RF)}
\vspace{0.5cm}
Los requerimientos funcionales describen las capacidades que el modelo debe tener:
\bigskip
\begin{table}[H]
    \centering
    \caption{Requerimientos Funcionales}
    \begin{adjustbox}{width=\textwidth}
    \begin{tabular}{p{1cm}p{5cm}p{1.75cm}p{2.25cm}p{5cm}}
    \hline %chktex 44
    \textbf{ID} & \textbf{Descripción} & \textbf{Prioridad} & \textbf{Actor} & \textbf{Criterios de Aceptación} \\
    \hline %chktex 44
    RF-001 & El sistema debe permitir la modificación dinámica de la masa de los cuerpos celestes durante la simulación. & Alta & Usuario/Modelo & El usuario puede modificar la masa de los cuerpos celestes en tiempo real y la simulación responde adecuadamente a estos cambios. \\
    \hline %chktex 44
    RF-002 & El sistema debe simular las interacciones gravitacionales entre dos cuerpos celestes. & Alta & Modelo & La simulación calcula y representa correctamente las fuerzas gravitacionales entre los cuerpos según las leyes físicas establecidas. \\
    \hline %chktex 44
    RF-003 & El sistema debe usar el método multipolar rápido (FMM) y el algoritmo de Barnes-Hut para optimizar los cálculos gravitacionales. & Media & Modelo & Los algoritmos se implementan correctamente y mejoran el rendimiento de los cálculos en comparación con métodos directos. \\
    \hline %chktex 44
    RF-004 & El sistema debe implementar algoritmos bioinspirados para ajustar dinámicamente los parámetros y encontrar valores idóneos que garanticen la estabilidad del sistema. & Alta & Modelo & Los algoritmos bioinspirados funcionan correctamente y logran mantener la estabilidad del sistema durante los cambios de masa. \\
    \hline %chktex 44
    RF-005 & El sistema debe incluir un módulo para visualizar gráficamente la evolución de los dos cuerpos en iteraciones limitadas. & Media & Usuario & La visualización muestra claramente las trayectorias y estados de los cuerpos celestes durante la simulación. \\
    \hline %chktex 44
    RF-006 & El sistema debe ofrecer una interfaz básica para modificar parámetros y ver resultados. & Media & Usuario & El usuario puede interactuar con la interfaz para ajustar parámetros y visualizar los cambios en la simulación. \\
    \hline %chktex 44
    \end{tabular}
    \end{adjustbox}
\end{table}
\bigskip
\subsection{Requerimientos No Funcionales (RNF)}
\vspace{0.5cm}
Los requerimientos no funcionales especifican las cualidades y restricciones del modelo, incluyendo los dos implícitos que me pediste añadir:\\
\bigskip
\begin{table}[H]
    \centering
    \caption{Requerimientos No Funcionales}
    \begin{adjustbox}{width=\textwidth}
    \begin{tabular}{p{1.2cm}p{5cm}p{1.75cm}p{2.25cm}p{5cm}}
    \hline %chktex 44
    \textbf{ID} & \textbf{Descripción} & \textbf{Prioridad} & \textbf{Actor} & \textbf{Criterios de Aceptación} \\
    \hline %chktex 44
    RNF-001 & El sistema debe optimizar la complejidad de las simulaciones de n-cuerpos para ser eficiente. & Alta & Modelo & La simulación se ejecuta en tiempos razonables sin consumir excesivos recursos computacionales. \\
    \hline %chktex 44
    RNF-002 & El sistema debe permitir la extensión futura a más de dos cuerpos. & Media & Modelo & La arquitectura del sistema es escalable y puede adaptarse para simular más cuerpos sin cambios estructurales mayores. \\
    \hline %chktex 44
    RNF-003 & El sistema debe respetar las leyes de la física y la mecánica celeste en las interacciones gravitacionales. & Alta & Modelo & Las simulaciones producen resultados coherentes con las leyes físicas establecidas. \\
    \hline %chktex 44
    RNF-004 & El sistema debe mantener simulaciones estables al modificar parámetros dinámicamente. & Alta & Modelo & La simulación no colapsa ni presenta comportamientos erráticos cuando se modifican los parámetros durante la ejecución. \\
    \hline %chktex 44
    RNF-005 & La interfaz debe ser intuitiva y accesible sin necesidad de conocimientos avanzados. & Media & Usuario & Usuarios sin experiencia técnica pueden utilizar el sistema sin dificultades significativas. \\
    \hline %chktex 44
    RNF-006 & El sistema debe ejecutarse en hardware de gama media (procesadores multinúcleo, 16 GB de RAM). & Media & Modelo & El sistema funciona correctamente en equipos con las especificaciones mínimas establecidas. \\
    \hline %chktex 44
    RNF-007 & El sistema debe integrarse con entornos virtuales como Unreal Engine. & Baja & Modelo & La integración con Unreal Engine u otros entornos similares es posible y funcional. \\
    \hline %chktex 44
    RNF-008 & El sistema debe ser modular para facilitar actualizaciones y correcciones. & Media & Desarrollador & Los componentes del sistema están claramente separados y pueden modificarse independientemente. \\
    \hline %chktex 44
    RNF-009 & El sistema debe incluir documentación clara para usuarios y desarrolladores. & Media & Desarrollador & La documentación es completa, comprensible y cubre todos los aspectos relevantes del sistema. \\
    \hline %chktex 44
    \end{tabular}
    \end{adjustbox}
    \end{table}