\section[FrameworkQt \& PyQt]{Framework de interfaz: Qt \& PyQt}

\subsection{Introducción a Qt}
El framework Qt se define formalmente como un entorno de desarrollo de aplicaciones multiplataforma, diseñado para la creación de interfaces gráficas de usuario (GUI) y aplicaciones que pueden ejecutarse en diversos sistemas operativos y plataformas de hardware, incluyendo Linux, Windows, macOS, Android y sistemas embebidos, con mínimas o nulas modificaciones en el código subyacente~\cite{qt_wiki}. Constituye un conjunto integral de clases de biblioteca C++ altamente intuitivas y modularizadas, enriquecidas con APIs que simplifican el proceso de desarrollo de aplicaciones~\cite{qt_framework}. Esta capacidad permite a los programadores escribir código una sola vez y distribuirlo a través de múltiples plataformas, abarcando desde aplicaciones de escritorio hasta sistemas embebidos~\cite{businesscloud}. La cualidad de ser multiplataforma, a menudo resumida en el principio de ``escribir una vez, ejecutar en cualquier lugar'' (WORA, por sus siglas en inglés), representa una ventaja fundamental de Qt, ya que reduce significativamente los tiempos y costos de desarrollo al mismo tiempo que permite alcanzar una audiencia más amplia~\cite{lemberg}. Además, el hecho de que C++ sea el lenguaje subyacente~\cite{qt_wiki} sugiere un enfoque en el rendimiento y el acceso a funcionalidades de bajo nivel del sistema.

La historia de Qt se remonta a 1990, cuando los programadores noruegos Eirik Chambe-Eng y Haavard Nord concibieron la idea, con el primer lanzamiento público en 1995 para el sistema operativo Linux~\cite{qt_history}. Inicialmente fue desarrollado por la empresa Trolltech, que posteriormente fue adquirida por Nokia, luego por Digia y actualmente es conocida como The Qt Company~\cite{qt_history}. Su popularidad experimentó un crecimiento significativo tras ser utilizado en la creación del entorno de escritorio KDE~\cite{qt_framework}. A lo largo de los años, Qt ha evolucionado desde una simple biblioteca de clases hasta convertirse en un framework extenso, experimentando cambios en sus modelos de licenciamiento~\cite{qt_framework}. Esta trayectoria de larga data y evolución sugiere un framework maduro y confiable, refinado a través de un uso extensivo y contribuciones continuas. El enfoque inicial en el soporte multiplataforma para Unix, Macintosh y Windows~\cite{qt_history} subraya el principio fundacional de la independencia de la plataforma.

La arquitectura de Qt se basa en el lenguaje de programación C++, el cual es extendido por un preprocesador denominado Meta-Object Compiler (moc)~\cite{qt_wiki}. El moc interpreta ciertas macros presentes en el código C++ como anotaciones y genera código C++ adicional que contiene meta-información sobre las clases utilizadas~\cite{qt_wiki}. Esta meta-información habilita características como \textit{signals} y \textit{slots}, introspección y llamadas a funciones asíncronas, las cuales no están disponibles de forma nativa en C++~\cite{qt_wiki}. El moc se erige como un componente arquitectónico clave que distingue a Qt al facilitar sus funcionalidades únicas para la comunicación entre objetos y el comportamiento dinámico. El hecho de que Qt extienda C++ en lugar de ser un lenguaje independiente~\cite{qt_wiki} permite a los desarrolladores aprovechar los beneficios de rendimiento de C++ al mismo tiempo que utilizan las abstracciones proporcionadas por Qt.

\subsection{Principios Fundamentales de Qt}
Qt presenta una arquitectura modular, compuesta por módulos esenciales y módulos complementarios~\cite{qt_wiki}. Los \textit{Qt Essentials} constituyen la base de Qt en todas las plataformas compatibles e incluyen módulos como \texttt{Qt Core} (funcionalidades no gráficas), \texttt{Qt GUI} (interfaz gráfica), \texttt{Qt Multimedia}, \texttt{Qt Network}, \texttt{Qt Quick} (interfaz de usuario declarativa) y \texttt{Qt SQL} (integración de bases de datos)~\cite{qt_framework}. Por otro lado, los \textit{Qt Add-Ons} ofrecen módulos para propósitos específicos que podrían no estar disponibles en todas las plataformas, como \texttt{Qt OpenGL}, \texttt{Qt Wayland Compositor}, \texttt{Qt Sensors}, \texttt{Qt WebView}, \texttt{Qt Safe Renderer} y \texttt{Qt SCXML}~\cite{qt_framework}. Esta modularidad permite a los desarrolladores incluir únicamente los componentes necesarios, lo que potencialmente reduce el tamaño de la aplicación~\cite{qt_framework}. La separación en ``Essentials'' y ``Add-Ons'' sugiere un diseño que atiende tanto a las necesidades de desarrollo comunes como a las especializadas.

\subsubsection{Signals y Slots}
Un principio fundamental de Qt es el mecanismo de \textit{signals} y \textit{slots}, una construcción del lenguaje introducida para la comunicación entre objetos. Este mecanismo facilita la implementación del patrón observador, evitando la necesidad de escribir código repetitivo. Un \textit{signal} es emitida por un objeto cuando su estado interno cambia de una manera que podría ser de interés para otros objetos~\cite{qt_mit_signals}. Un \textit{slot} es una función que se llama en respuesta a una \textit{signal} particular~\cite{qt_mit_signals}. Las firmas de una \textit{signal} deben coincidir con la firma del \textit{slot} receptor, garantizando la seguridad de tipos~\cite{qt_mit_signals}. Un \textit{slot} puede tener una firma más corta al ignorar argumentos adicionales~\cite{qt_mit_signals}. Los \textit{signals} y \textit{slots} están débilmente acoplados; el objeto emisor no necesita conocer qué \textit{slots} recibirán la \textit{signal}~\cite{qt_mit_signals}. Las conexiones entre \textit{signals} y \textit{slots} se establecen mediante la función \texttt{connect~()}~\cite{qt_mit_signals}. Se pueden conectar múltiples \textit{signals} a un solo \textit{slot}, y una sola \textit{signal} se puede conectar a múltiples \textit{slots}~\cite{qt_mit_signals}. Este mecanismo de \textit{signals} y \textit{slots} representa una innovación central de Qt que simplifica el manejo de eventos y la comunicación entre componentes de una manera segura en cuanto a tipos y desacoplada. Resulta particularmente adecuado para la programación de GUIs, donde las interacciones del usuario desencadenan comportamientos específicos~\cite{businesscloud}.

\subsubsection{Widgets y Layouts}
Para la construcción de interfaces gráficas, Qt introduce los conceptos de \textit{widgets} y \textit{layouts}. Los programas Qt se crean utilizando partes denominadas \textit{widgets}, que son los bloques de construcción básicos de cualquier interfaz de usuario, abarcando desde entradas de texto y etiquetas hasta ventanas y botones~\cite{businesscloud}. Qt utiliza \textit{layouts} para posicionar estos \textit{widgets} en la pantalla, controlando automáticamente su tamaño y ubicación para adaptarse a diferentes tamaños y resoluciones de pantalla~\cite{businesscloud}. Los administradores de \textit{layouts} comunes incluyen \texttt{QVBoxLayout}, \texttt{QHBoxLayout} y \texttt{QGridLayout}~\cite{tutorialspoint_pyqt}. El sistema de \textit{widgets} y \textit{layouts} proporciona una manera estructurada y consistente en todas las plataformas para diseñar interfaces de usuario. Los \textit{layouts} son cruciales para crear aplicaciones responsivas que se adaptan a diversas dimensiones de pantalla.
Qt también ofrece \texttt{Qt Quick}, un framework declarativo para construir aplicaciones altamente dinámicas con interfaces de usuario personalizadas~\cite{qt_wiki}. Las GUIs que utilizan \texttt{Qt Quick} se escriben en QML (Qt Meta Language o Qt Modeling Language), un lenguaje de descripción de objetos declarativo que integra JavaScript para la programación procedimental~\cite{qt_wiki}.\ \texttt{Qt Quick} facilita el desarrollo rápido de aplicaciones, especialmente para dispositivos móviles, con la posibilidad de escribir lógica en código nativo (C++) para las partes que requieren un rendimiento crítico~\cite{qt_wiki}.\ \texttt{Qt Quick} y QML ofrecen un enfoque más moderno y a menudo más eficiente en términos de rendimiento para el desarrollo de interfaces de usuario, particularmente para interfaces táctiles y animaciones~\cite{qt_for_python}. La separación del diseño de la interfaz de usuario (QML) y la lógica de backend (C++ o Python con PyQt) promueve una mejor organización del código y colaboración entre diseñadores y desarrolladores~\cite{qt_framework}.

\subsection{Variantes y Licencias de Qt}
Qt está disponible bajo licencias tanto comerciales como de código abierto. Las licencias de código abierto incluyen diversas versiones de la GPL (GNU General Public License) y la LGPL (GNU Lesser General Public License)~\cite{qt_wiki}. Las licencias comerciales son ofrecidas por The Qt Company y proporcionan características adicionales, soporte y opciones de licenciamiento adecuadas para el desarrollo de software propietario~\cite{qt_wiki}. Este modelo de doble licencia permite que Qt se utilice en una amplia gama de proyectos, desde código abierto hasta aplicaciones comerciales. Comprender los términos de la licencia es crucial para los desarrolladores para garantizar el cumplimiento de los requisitos de su proyecto~\cite{qt_wiki}.
Las versiones principales de Qt incluyen Qt 5 (lanzado en 2012) y Qt 6 (lanzado en 2020)~\cite{qt_framework}. Qt 5 introdujo cambios significativos como una nueva base de código modularizada, la consolidación de QPA (Qt Platform Abstraction) y Qt Quick 2~\cite{qt_history}. Qt 6 representa una evolución adicional con actualizaciones y mejoras continuas~\cite{qt_framework}. Las versiones de soporte a largo plazo (LTS) están disponibles para usuarios comerciales, proporcionando estabilidad y períodos de soporte extendidos~\cite{qt_wiki}. La existencia de versiones principales indica un desarrollo y evolución continuos del framework, con cada versión introduciendo nuevas funcionalidades y mejoras. La disponibilidad de versiones LTS es importante para proyectos que requieren estabilidad y soporte a largo plazo, como aplicaciones industriales o críticas para la seguridad~\cite{lemberg}.

\subsection{Aplicaciones Típicas de Qt}
Qt se utiliza en el desarrollo de aplicaciones que se ejecutan en las principales plataformas de escritorio (Windows, macOS, Linux), plataformas móviles (Android, iOS) y plataformas embebidas~\cite{qt_wiki}. Es una opción popular para crear aplicaciones multiplataforma con interfaces de aspecto nativo~\cite{qt_wiki}. Qt también se utiliza ampliamente en el mercado de sistemas embebidos debido a su soporte para diversos sistemas operativos embebidos como Linux y QNX~\cite{lemberg}. La amplia compatibilidad con plataformas convierte a Qt en una opción versátil para diversos dominios de aplicación. Su idoneidad para sistemas embebidos resalta su eficiencia y adaptabilidad a entornos con recursos limitados~\cite{qt_framework}.
Qt se emplea en diversas industrias, incluyendo la automotriz (tableros digitales, sistemas de infoentretenimiento), la salud (instrumentos médicos), la aeroespacial y defensa, la automatización industrial y la electrónica de consumo~\cite{lemberg}. Ejemplos de aplicaciones conocidas construidas con Qt incluyen Google Earth, Skype, VirtualBox, Autodesk Maya, Telegram y partes de la interfaz KDE~\cite{qt_use_cases}. En el ámbito de la computación científica, PyQt (el binding de Python para Qt) se utiliza para construir herramientas de análisis y visualización de datos, así como interfaces gráficas de usuario para equipos de laboratorio y sistemas de adquisición de datos~\cite{ontosight_pyqt_basics}.\ \texttt{Qt Quick 3D Physics} proporciona una API de alto nivel para la simulación física, soportando cuerpos rígidos y mallas estáticas~\cite{qt_quick}. Herramientas como QTCAD® están construidas con Qt para simular hardware de tecnología cuántica~\cite{ontosight_pyqt_basics}. La amplia gama de aplicaciones demuestra la versatilidad y potencia del framework Qt en diferentes sectores, incluyendo aquellos relevantes para la investigación científica y la simulación. La mención específica de Qt en la simulación física y la visualización científica subraya su potencial utilidad en el contexto del proyecto principal.

\subsection{Ventajas y Desventajas de Qt}
\begin{itemize}
    \item \textbf{Ventajas}: Entre las fortalezas inherentes a Qt se destaca su compatibilidad multiplataforma, que permite a los desarrolladores escribir código una vez y desplegarlo en múltiples sistemas operativos, ahorrando tiempo y recursos~\cite{qt_wiki}. Su rico conjunto de herramientas y bibliotecas ofrece una amplia gama de funcionalidades, desde elementos de interfaz de usuario hasta redes, multimedia y gráficos 3D~\cite{qt_framework}. Al estar basado en C++, Qt proporciona un alto rendimiento nativo y un tamaño reducido~\cite{qt_framework}. Además, cuenta con una gran y activa comunidad que ofrece tutoriales, foros y bibliotecas para soporte y aprendizaje~\cite{businesscloud}. Su documentación exhaustiva incluye tutoriales detallados, ejemplos y referencias de la API~\cite{lemberg}. Estas ventajas convierten a Qt en un framework potente y eficiente para desarrollar aplicaciones complejas en diversas plataformas.
    \item \textbf{Desventajas}: Sin embargo, Qt también presenta ciertas limitaciones. Su curva de aprendizaje puede ser pronunciada para principiantes, especialmente para aquellos que no están familiarizados con C++ o la programación de interfaces gráficas, particularmente en la comprensión de conceptos como \textit{signals} y \textit{slots}~\cite{itsupplychain}. Los costos de las licencias comerciales pueden ser elevados, lo que podría ser una preocupación para startups o proyectos más pequeños~\cite{qt_wiki}. No todas las partes de Qt están bajo la licencia LGPL~\cite{lemberg}. Para aplicaciones pequeñas o simples, Qt podría percibirse como demasiado pesado en comparación con frameworks más sencillos~\cite{itsupplychain}. La capa de abstracción en Qt podría introducir cierta sobrecarga de rendimiento, lo que podría ser una preocupación para sistemas embebidos con recursos muy limitados~\cite{lemberg}. Si bien Qt ofrece ventajas significativas, los desarrolladores deben ser conscientes de su curva de aprendizaje y las implicaciones de la licencia, especialmente para proyectos comerciales. La percepción de que es demasiado pesado para tareas simples podría llevar a algunos a elegir frameworks alternativos.
\end{itemize}

\subsection{Introducción a PyQt}
PyQt se define como un conjunto de bindings de Python para el framework de aplicaciones Qt de The Qt Company~\cite{pyqt_wiki}. Estos bindings se implementan como un conjunto de módulos de Python y contienen un gran número de clases, superando las 1000~\cite{pyqt_wiki}. PyQt actúa como un puente, integrando de manera fluida el robusto framework multiplataforma Qt C++ con el lenguaje de programación Python, conocido por su flexibilidad~\cite{qt_for_python}. Esto permite a los desarrolladores de Python aprovechar la potencia y las características del framework Qt utilizando la sintaxis de Python. El extenso número de clases disponibles en PyQt indica un acceso integral a las funcionalidades de Qt.
PyQt fue lanzado por primera vez en 1998 por la firma británica Riverbank Computing~\cite{pyqt_wiki}. Fue desarrollado por Phil Thompson utilizando la herramienta SIP para generar automáticamente los bindings~\cite{pyqt_wiki}. Posteriormente, Nokia (entonces propietaria de Qt) lanzó sus propios bindings de Python denominados PySide (ahora Qt for Python) debido a desacuerdos en las licencias~\cite{pyqt_wiki}. El desarrollo temprano de PyQt resalta el deseo de llevar las capacidades de Qt al ecosistema de Python. La existencia de PySide como un binding alternativo proporciona a los desarrolladores más opciones de licencia~\cite{tutorialspoint_pyqt}.
Qt for Python es el conjunto oficial de bindings de Python (PySide6) respaldado por The Qt Company~\cite{qt_for_python}. PyQt y PySide ofrecen funcionalidades similares ya que ambos se construyen sobre Qt~\cite{tutorialspoint_pyqt}. Una diferencia clave radica a menudo en la licencia, siendo PySide típicamente disponible bajo la LGPL, que es más permisiva para proyectos comerciales en comparación con las licencias GPL y comerciales de PyQt~\cite{tutorialspoint_pyqt}. Si bien PyQt fue el primer binding de Python para Qt ampliamente adoptado, Qt for Python (PySide) es ahora la oferta oficial de los mantenedores de Qt. La elección entre PyQt y PySide a menudo depende de las necesidades específicas de licencia del proyecto.

\subsection{Principios Fundamentales de PyQt}
PyQt integra la funcionalidad de Qt en el entorno de programación Python proporcionando acceso a la funcionalidad de Qt a través de módulos de extensión de Python~\cite{tutorialspoint_pyqt}. Los desarrolladores pueden importar estos módulos (por ejemplo, \texttt{QtCore}, \texttt{QtGui}, \texttt{QtWidgets}) para utilizar las clases y funciones de Qt dentro de su código Python~\cite{tutorialspoint_pyqt}. PyQt conserva gran parte de la sintaxis de Qt, lo que facilita a los desarrolladores familiarizados con Qt/C++ la transición a Python~\cite{qt_for_python}. PyQt actúa como un envoltorio alrededor de la biblioteca Qt C++, exponiendo sus características a Python. La estrecha semejanza de sintaxis entre PyQt y Qt/C++ puede facilitar el intercambio y la comprensión del código entre lenguajes.
PyQt refleja la estructura modular de Qt, proporcionando módulos de Python que corresponden a los módulos C++ de Qt (por ejemplo, \texttt{QtCore} para funcionalidades centrales no gráficas, \texttt{QtGui} para componentes de GUI, \texttt{QtNetwork} para redes, \texttt{QtSql} para acceso a bases de datos)~\cite{tutorialspoint_pyqt}. Cada módulo contiene clases de Python que proporcionan acceso a las clases C++ de Qt correspondientes y sus funcionalidades~\cite{tutorialspoint_pyqt}. Esta modularidad permite a los desarrolladores de Python importar y utilizar selectivamente solo las partes necesarias del framework Qt.
PyQt implementa el mecanismo de \textit{signals} y \textit{slots} de Qt en Python, permitiendo que los objetos de Python se comuniquen de una manera segura en cuanto a tipos y desacoplada~\cite{businesscloud}. Los métodos de Python se pueden definir como \textit{slots} y conectarse a \textit{signals} de Qt, habilitando la programación orientada a eventos~\cite{businesscloud}. Las \textit{signals} son emitidas por los \textit{widgets} de Qt y se pueden conectar a \textit{slots} de Python para desencadenar acciones en respuesta a interacciones del usuario u otros eventos~\cite{businesscloud}. La integración fluida de \textit{signals} y \textit{slots} en Python permite construir aplicaciones GUI interactivas y responsivas.
PyQt se integra con Qt Designer, una herramienta de construcción de GUI visual proporcionada por Qt~\cite{businesscloud}. Los desarrolladores pueden diseñar interfaces de usuario utilizando una interfaz de arrastrar y soltar en Qt Designer y luego utilizar PyQt para cargar estos diseños en sus aplicaciones Python~\cite{lemberg}. PyQt puede generar código Python a partir de archivos \texttt{.ui} de Qt Designer utilizando herramientas como \texttt{pyuic}~\cite{tutorialspoint_pyqt}. La integración con Qt Designer facilita la creación rápida de prototipos y el desarrollo visual de GUIs en Python.

\subsection{Aplicaciones Típicas de PyQt}
PyQt se utiliza principalmente para crear programas con interfaz gráfica de usuario en Python, siendo reconocido por su facilidad de uso, flexibilidad y la apariencia nativa de sus aplicaciones GUI~\cite{tutorialspoint_pyqt}. Es una opción popular para construir aplicaciones de escritorio multiplataforma, aplicaciones móviles y sistemas embebidos~\cite{ontosight_pyqt_basics}. Su versatilidad y naturaleza multiplataforma lo convierten en una herramienta valiosa para el desarrollo de interfaces gráficas en Python.
En el ámbito de la computación científica, PyQt se utiliza para construir herramientas de análisis y visualización de datos, como interfaces gráficas para equipos de laboratorio y sistemas de adquisición de datos~\cite{ontosight_pyqt_basics}. También se aplica en la creación de aplicaciones para el análisis y visualización de imágenes médicas, como visores de resonancias magnéticas y tomografías computarizadas~\cite{ontosight_pyqt_basics}. Además, se utiliza en herramientas para la gestión de proyectos, la visión por computador y el análisis de datos~\cite{ontosight_pyqt_basics}. Su integración con bibliotecas como PyQtGraph y Matplotlib permite realizar gráficos y visualizaciones avanzadas~\cite{qt_for_python}. Dada su aplicación en la computación científica y la visualización de datos, PyQt podría ser una herramienta valiosa para la creación de interfaces de usuario para simulaciones físicas, permitiendo el control interactivo y la visualización de los parámetros y resultados de la simulación. El framework Qt subyacente también cuenta con módulos como \texttt{Qt Quick 3D Physics}, lo que sugiere un potencial para capacidades de simulación física más integradas, aunque PyQt principalmente proporciona la capa de interfaz en Python.

\subsection{Ventajas y Desventajas de PyQt}
\begin{itemize}
    \item \textbf{Ventajas}: Entre las ventajas de utilizar PyQt se encuentra la facilidad de uso y flexibilidad que ofrece la combinación de la simplicidad de Python con las potentes características de Qt, lo que lo hace relativamente fácil de aprender y utilizar para el desarrollo de interfaces gráficas~\cite{ontosight_pyqt_basics}. Su compatibilidad multiplataforma permite que las aplicaciones PyQt se ejecuten en Windows, macOS, Linux y otras plataformas sin modificaciones significativas en el código~\cite{lemberg}. Además, PyQt proporciona extensas bibliotecas y herramientas para la construcción de aplicaciones GUI complejas, incluyendo \textit{widgets}, gráficos, soporte de red, soporte de bases de datos y multimedia~\cite{tutorialspoint_pyqt}. La integración con Qt Designer permite un desarrollo rápido de interfaces de usuario a través de un diseño visual~\cite{lemberg}. Finalmente, cuenta con una gran y activa comunidad que proporciona recursos, soporte y facilita la resolución de problemas~\cite{lemberg}.
    \item \textbf{Desventajas}: Sin embargo, también presenta desventajas. Su licencia es dual, bajo GPL v3 y una licencia comercial. Para aplicaciones comerciales donde no se desea una licencia compatible con GPL, se debe adquirir una licencia comercial~\cite{tutorialspoint_pyqt}. A diferencia de Qt, PyQt no está disponible bajo la LGPL~\cite{pyqt_wiki}. Al ser un wrapper alrededor de C++, las aplicaciones PyQt podrían ser susceptibles a errores críticos que pueden provocar el cierre inesperado del intérprete de Python~\cite{gfg_pyqt_vs_tkinter}. Su curva de aprendizaje puede ser más pronunciada en comparación con frameworks GUI puramente basados en Python como Tkinter~\cite{itsupplychain}. La documentación podría estar más centrada en el framework Qt en C++, lo que requeriría cierta interpretación para los desarrolladores de Python~\cite{gfg_pyqt_vs_tkinter}. En algunos casos, los desarrolladores podrían necesitar gestionar explícitamente las actualizaciones de la interfaz de usuario utilizando \textit{signals} y \textit{slots}, lo que puede volverse complejo en aplicaciones grandes~\cite{gfg_pyqt_vs_tkinter}.
\end{itemize}