\section{Análisis de Procesos Internos del Programa}

El análisis detallado de los procesos internos del software constituye un pilar esencial para el diseño, implementación y validación del modelo de simulación gravitacional propuesto en este proyecto. Comprender la secuencia de operaciones, el flujo de datos y la interacción entre los componentes clave (como la biblioteca \texttt{REBOUND}, la lógica de modificación y la interfaz de usuario) es fundamental para abordar el desafío central del proyecto: habilitar la modificación dinámica de parámetros, como la masa, en un sistema de N-cuerpos (inicialmente abordado con dos cuerpos) durante la ejecución de la simulación.

Esta sección establece la base operativa sobre la cual se construirá la solución. Proporciona el mapa necesario para implementar las innovaciones clave del proyecto –la modificación dinámica de parámetros– de manera estructurada, eficiente y verificable, asegurando que el modelo final cumpla con el objetivo de superar las limitaciones de las herramientas de simulación actuales.

\subsection{Procesos Internos del Programa y sus Actores}

Los siguientes procesos describen el funcionamiento interno del programa para simular comportamientos gravitacionales de dos cuerpos, utilizando la biblioteca \texttt{REBOUND}. Cada proceso incluye sus actores, que son los componentes responsables de ejecutar las acciones.

\begin{enumerate}
    \item \textbf{Configuración Inicial de la Simulación}
        \begin{itemize}
            \item \textbf{Objetivo:} Preparar la simulación estableciendo los parámetros iniciales de los dos cuerpos celestes y las condiciones del sistema.
            \item \textbf{Actores:} Biblioteca \texttt{REBOUND}.
                \begin{itemize}
                    \item \texttt{REBOUND} crea la instancia de simulación y configura los cuerpos celestes con sus parámetros iniciales.
                \end{itemize}
            \item \textbf{Inicio:} Lanzamiento del programa o inicio de una nueva simulación.
            \item \textbf{Entradas:}
                \begin{itemize}
                    \item Masa, posición y velocidad inicial de los dos cuerpos.
                    \item Constante gravitacional y otros parámetros físicos.
                \end{itemize}
            \item \textbf{Salidas:} Una simulación configurada en \texttt{REBOUND} lista para ejecutarse.
            \item \textbf{Pasos Clave:}
                \begin{enumerate}
                    \item Crear una instancia de simulación en \texttt{REBOUND}.
                    \item Agregar los dos cuerpos celestes con sus parámetros iniciales.
                    \item Definir las condiciones iniciales del sistema (tiempo, paso de integración, etc.).
                \end{enumerate}
        \end{itemize}

    \item \textbf{Ejecución de la Simulación Gravitacional}
        \begin{itemize}
            \item \textbf{Objetivo:} Simular las interacciones gravitacionales entre los dos cuerpos utilizando los algoritmos integrados en \texttt{REBOUND}.
            \item \textbf{Actores:} Biblioteca \texttt{REBOUND}.
                \begin{itemize}
                    \item \texttt{REBOUND} utiliza sus integradores numéricos para resolver las ecuaciones de movimiento y actualizar el estado del sistema.
                \end{itemize}
            \item \textbf{Inicio:} Comando para ejecutar la simulación tras la configuración.
            \item \textbf{Entradas:} Configuración inicial generada en el proceso anterior.
            \item \textbf{Salidas:} Datos temporales de la simulación (posiciones, velocidades, etc.).
            \item \textbf{Pasos Clave:}
                \begin{enumerate}
                    \item Iniciar la simulación mediante \texttt{REBOUND}.
                    \item Utilizar los integradores de \texttt{REBOUND} para resolver las ecuaciones de movimiento.
                    \item Actualizar el estado del sistema (posiciones y velocidades) en cada paso de tiempo.
                \end{enumerate}
        \end{itemize}

    \item \textbf{Modificación Dinámica de Parámetros}
        \begin{itemize}
            \item \textbf{Objetivo:} Ajustar parámetros como la masa de los cuerpos durante la simulación para observar cambios en el comportamiento gravitacional.
            \item \textbf{Actores:}
                \begin{itemize}
                    \item Biblioteca \texttt{REBOUND}: Actualiza los parámetros dentro de la simulación.
                    \item Lógica interna del programa: Gestiona la pausa y reanudación de la simulación para aplicar los cambios.
                \end{itemize}
            \item \textbf{Inicio:} Evento definido (manual o automático) que requiere ajustar parámetros.
            \item \textbf{Entradas:} Nuevos valores para parámetros (ej.\ masa actualizada).
            \item \textbf{Salidas:} Simulación actualizada con los nuevos parámetros.
            \item \textbf{Pasos Clave:}
                \begin{enumerate}
                    \item Detectar la necesidad de modificar un parámetro.
                    \item Pausar o preparar la simulación para el cambio.
                    \item Actualizar los parámetros en \texttt{REBOUND}.
                    \item Continuar la simulación con los valores ajustados.
                \end{enumerate}
        \end{itemize}

    \item \textbf{Visualización de la Simulación}
        \begin{itemize}
            \item \textbf{Objetivo:} Mostrar gráficamente la evolución del sistema de dos cuerpos durante la simulación.
            \item \textbf{Actores:}
                \begin{itemize}
                    \item Módulo de visualización (herramienta gráfica específica): Procesa y representa los datos en forma de gráficos o animaciones.
                    \item Datos generados por \texttt{REBOUND}: Proporcionan la información que se visualiza.
                \end{itemize}
            \item \textbf{Inicio:} Necesidad de observar los resultados visualmente.
            \item \textbf{Entradas:} Datos generados por la simulación (posiciones, velocidades).
            \item \textbf{Salidas:} Representaciones gráficas o animaciones del sistema.
            \item \textbf{Pasos Clave:}
                \begin{enumerate}
                    \item Recolectar datos de la simulación en cada paso.
                    \item Procesar los datos para adaptarlos a la herramienta de visualización.
                    \item Generar gráficos o animaciones del movimiento gravitacional.
                \end{enumerate}
        \end{itemize}

    \item \textbf{Interacción con la Interfaz de Usuario}
        \begin{itemize}
            \item \textbf{Objetivo:} Permitir al usuario interactuar con la simulación, modificando parámetros y observando los resultados en tiempo real.
            \item \textbf{Actores:}
                \begin{itemize}
                    \item Interfaz de usuario: Captura las entradas del usuario (ej.\ clics, valores ingresados).
                    \item Módulo de simulación con \texttt{REBOUND}: Ejecuta las modificaciones basadas en las entradas del usuario.
                \end{itemize}
            \item \textbf{Inicio:} Acción del usuario (ej.\ clic en un botón o entrada de datos).
            \item \textbf{Entradas:} Instrucciones del usuario (ej.\ cambiar masa, iniciar/pausar simulación).
            \item \textbf{Salidas:} Simulación y visualización actualizadas según las acciones del usuario.
            \item \textbf{Pasos Clave:}
                \begin{enumerate}
                    \item Capturar las entradas del usuario a través de la interfaz.
                    \item Convertir las entradas en comandos para \texttt{REBOUND} o el módulo de visualización.
                    \item Ejecutar las modificaciones y reflejar los cambios en la simulación.
                \end{enumerate}
        \end{itemize}
\end{enumerate}

\section{Matriz de Procesos}
\begin{table}[H]
    \centering
    \caption{Matriz resumen de los procesos internos del programa.}
    \label{tab:matriz_procesos}
    \begin{adjustbox}{max width=\textwidth}
        \begin{tabular}{@{}p{5cm} p{3cm} p{2.5cm} p{2.5cm} p{2.5cm} p{3cm} p{4cm}@{}}
            \toprule
            \textbf{Nombre del Proceso} & \textbf{Objetivo} & \textbf{Actores} & \textbf{Inicio} & \textbf{Entradas} & \textbf{Salidas} & \textbf{Pasos Clave} \\
            \midrule
            \textbf{Configuración Inicial de la Simulación} & Preparar la simulación estableciendo los parámetros iniciales de los dos cuerpos celestes y las condiciones del sistema. & Biblioteca \texttt{REBOUND} & Lanzamiento del programa o inicio de una nueva simulación & Masa, posición y velocidad inicial de los dos cuerpos; constante gravitacional y otros parámetros físicos & Una simulación configurada en \texttt{REBOUND} lista para ejecutarse & 1. Crear una instancia de simulación en \texttt{REBOUND} \newline 2. Agregar los dos cuerpos celestes con sus parámetros iniciales \newline 3. Definir las condiciones iniciales del sistema \\
            \midrule
            \textbf{Ejecución de la Simulación Gravitacional} & Simular las interacciones gravitacionales entre los dos cuerpos utilizando los algoritmos integrados en \texttt{REBOUND}. & Biblioteca \texttt{REBOUND} & Comando para ejecutar la simulación tras la configuración & Configuración inicial generada en el proceso anterior & Datos temporales de la simulación & 1. Iniciar la simulación mediante \texttt{REBOUND} \newline 2. Utilizar los integradores de \texttt{REBOUND} para resolver las ecuaciones de movimiento \newline 3. Actualizar el estado del sistema en cada paso de tiempo \\
            \midrule
            \textbf{Modificación Dinámica de Parámetros} & Ajustar parámetros como la masa de los cuerpos durante la simulación para observar cambios en el comportamiento gravitacional. & Biblioteca \texttt{REBOUND}, lógica interna del programa & Evento definido que requiere ajustar parámetros & Nuevos valores para parámetros & Simulación actualizada con los nuevos parámetros & 1. Detectar la necesidad de modificar un parámetro \newline 2. Pausar o preparar la simulación para el cambio \newline 3. Actualizar los parámetros en \texttt{REBOUND} \newline 4. Continuar la simulación con los valores ajustados \\
            \midrule
            \textbf{Visualización de la Simulación} & Mostrar gráficamente la evolución del sistema de dos cuerpos durante la simulación. & Módulo de visualización, datos generados por \texttt{REBOUND} & Necesidad de observar los resultados visualmente & Datos generados por la simulación & Representaciones gráficas o animaciones del sistema & 1. Recolectar datos de la simulación en cada paso \newline 2. Procesar los datos para adaptarlos a la herramienta de visualización \newline 3. Generar gráficos o animaciones del movimiento gravitacional \\
            \midrule
            \textbf{Interacción con la Interfaz de Usuario} & Permitir al usuario interactuar con la simulación, modificando parámetros y observando los resultados en tiempo real. & Interfaz de usuario, módulo de simulación con \texttt{REBOUND} & Acción del usuario & Instrucciones del usuario & Simulación y visualización actualizadas según las acciones del usuario & 1. Capturar las entradas del usuario a través de la interfaz \newline 2. Convertir las entradas en comandos para \texttt{REBOUND} o el módulo de visualización \newline 3. Ejecutar las modificaciones y reflejar los cambios en la simulación \\
            \bottomrule
        \end{tabular}
    \end{adjustbox}
\end{table}

