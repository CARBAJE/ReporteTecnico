\subsection[Proceso Interno 08: Resolver Ecuaciones]{Proceso Interno 08: Avanzar un Paso de Simulación Usando al Integrador WHFast (Resolver Ecuaciones)}

\subsubsection{Objetivo del Proceso}
Actualizar el estado del sistema gravitacional usando el esquema Drift-Kick-Drift (DKD) de WHFast:
\[
H = H_{\text{Kepler}} + H_{\text{Interaction}}
\]
Garantizando precisión en el cálculo de posiciones y velocidades para la evaluación del LE.\

\subsubsection{Entradas Principales}
\begin{itemize}
    \item \textbf{sim}: Objeto \texttt{Simulation} con:
    \begin{itemize}
        \item $\mathbf{r}_i(t) = [x_i, y_i, z_i]$
        \item $\mathbf{v}_i(t) = [v_{x_i}, v_{y_i}, v_{z_i}]$
        \item $m_i$, $dt$, sistema coordenadas
    \end{itemize}
    \item $t_{\text{target}} = t + dt$
\end{itemize}

\subsubsection{Sub-pasos Secuenciales}
Este apartado es proporcionado para profundizar y describir de forma textual cada paso contenido dentro del diagrama del proceso descrito en la figura~\ref{fig:process_diagram08}
\subsubsection*{1. Preparación de Coordenadas}
\begin{itemize}
    \item Transformación a coordenadas Jacobi si es necesario:
    \[
    \mathbf{r}_{\text{Jacobi}} = T(\mathbf{r}_{\text{inercial}})
    \]
    \item Verificación modo seguro
\end{itemize}

\subsubsection*{2. Primer Drift ($dt/2$)}
\begin{itemize}
    \item Integración Kepleriana:
    \[
    E - e\sin E = M \quad \text{(Ecuación de Kepler)}
    \]
    \item Solucionador mejorado (Newton/Laguerre-Conway)
    \item Actualización posiciones: $\mathbf{r}_i(t + dt/2)$
\end{itemize}

\subsubsection*{3. Transformación para Interacción}
\begin{itemize}
    \item Conversión a coordenadas baricéntricas si es necesario
\end{itemize}

\subsubsection*{4. Kick ($dt$)}
\begin{itemize}
    \item Cálculo de aceleraciones:
    \[
    \mathbf{a}_i = G\sum_{j\neq i}\frac{m_j(\mathbf{r}_j - \mathbf{r}_i)}{|\mathbf{r}_j - \mathbf{r}_i|^3}
    \]
    \item Actualización velocidades:
    \[
    \mathbf{v}_i(t + dt) = \mathbf{v}_i(t) + \mathbf{a}_i \cdot dt
    \]
    \item Kernel modificado (si aplica)
\end{itemize}

\subsubsection*{5. Segundo Drift ($dt/2$)}
\begin{itemize}
    \item Integración Kepleriana final
    \item Posiciones finales: $\mathbf{r}_i(t + dt)$
\end{itemize}

\subsubsection*{6. Sincronización Final}
\begin{itemize}
    \item Transformación a coordenadas de reporte
    \item Actualización tiempo: $\texttt{sim.t} = t_{\text{target}}$
    \item Verificaciones modo seguro
\end{itemize}

\subsubsection{Lógica Interna y Decisiones}
\begin{itemize}
    \item \textbf{Transformaciones coordenadas}:
    \begin{itemize}
        \item Optimizadas para precisión numérica
        \item Jacobi $\leftrightarrow$ Baricéntricas
    \end{itemize}
    \item \textbf{Modo seguro}: Verificaciones de estabilidad
    \item \textbf{Kernel}: Selección entre precisión/rendimiento
\end{itemize}

\subsubsection{Manejo de Datos Específico}
\begin{itemize}
    \item \textbf{Entradas}: Estado del sistema en $t$
    \item \textbf{Intermedios}:
    \begin{itemize}
        \item Estados parciales DKD
        \item Coordenadas transformadas
    \end{itemize}
    \item \textbf{Salida}: Estado del sistema en $t + dt$
\end{itemize}

\subsubsection{Salidas Principales}
\begin{itemize}
    \item \texttt{sim} actualizado con:
    \begin{itemize}
        \item $\mathbf{r}_i(t + dt)$
        \item $\mathbf{v}_i(t + dt)$
        \item $t = t + dt$
    \end{itemize}
\end{itemize}

\subsubsection{Interacciones Internas}
\begin{itemize}
    \item \textbf{API REBOUND}:
    \begin{itemize}
        \item Módulo WHFast y solucionador Kepler
        \item Transformaciones coordenadas
    \end{itemize}
    \item \textbf{Gestor de memoria}: Almacenamiento estados intermedios
    \item \textbf{Sistema físico}: Conservación propiedades dinámicas
\end{itemize}
\newpage
\subsubsection{Diagrama del Proceso}
\begin{figure}[H]
    \centering
    \adjustbox{max width=\textwidth, max height=0.9\textheight}{%
        \includegraphics{img/Analisis/DiagramaProcesos/DiagramaProceso08_ResolverEcuaciones.png}
    }
        \caption{Diagrama de Proceso Interno 08: Resolver Ecuaciones}%
    \label{fig:process_diagram08}
\end{figure}
\newpage