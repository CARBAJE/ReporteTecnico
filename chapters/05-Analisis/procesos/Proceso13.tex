\subsection{Proceso Interno 13: Evaluar Fitness}

\subsubsection{Objetivo del Proceso}
El propósito principal es calcular el \textit{fitness penalizado} $F_p(x) = f(x) + \lambda_p P(x)$, donde:
\begin{itemize}
    \item $f(x)$: Exponente de Lyapunov ($LE$)
    \item $P(x) = \sum {\max(g_k(x), 0)}^2 + \sum |h_k(x)|^2$
    \item $\lambda_p$: Parámetro de penalización
\end{itemize}

\subsubsection{Entradas Principales}
\begin{itemize}
    \item \textbf{Parámetros $x$}: Vector en $\mathbb{R}^n$ (ej. $[m_1, m_2]$)
    \item \textbf{Datos REBOUND}: ${\{\text{pos}(t), \text{vel}(t), \text{mass}(t)\}}_{t=0}^{T_{\max}}$
    \item \textbf{Restricciones}:
    \begin{itemize}
        \item $g_k(x) \leq 0$ (desigualdad)
        \item $h_k(x) = 0$ (igualdad)
    \end{itemize}
    \item $\lambda_p \in \mathbb{R}^+$
\end{itemize}

\subsubsection{Sub-pasos Secuenciales}
Este apartado es proporcionado para profundizar y describir de forma textual cada paso contenido dentro del diagrama del proceso descrito en la figura~\ref{fig:process_diagram13}
\subsubsection*{1. Recibir Resultados de Simulación}
\begin{itemize}
    \item Validar integridad de datos y $T_{\max}$
\end{itemize}

\subsubsection*{2. Calcular $f(x)$}
\begin{itemize}
    \item Cálculo de $LE$:
    \begin{equation*}
        f(x) = \begin{cases}
        LE_{\text{calc}} & \text{si converge} \\
        \texttt{INF} & \text{si falla}
        \end{cases}
    \end{equation*}
\end{itemize}

\subsubsection*{3. Calcular $P(x)$}
\begin{align*}
    P(x) &= \sum_{k=1}^{K} {\max(g_k(x), 0)}^2 + \sum_{j=1}^{J} {|h_j(x)|}^2 \\
    &+ \begin{cases}
    \texttt{PENALTY\_LARGE} & \text{si simulación inviable}
    \end{cases}
\end{align*}

\subsubsection*{4. Calcular $F_p(x)$}
\begin{equation*}
    F_p(x) = f(x) + \lambda_p \cdot P(x)
\end{equation*}
\begin{itemize}
    \item Verificar $F_p \in \mathbb{R}$
\end{itemize}

\subsubsection{Lógica Interna y Decisiones}
\begin{itemize}
    \item \textbf{Convergencia LE}:
    \begin{itemize}
        \item Fallo $\Rightarrow f(x) \leftarrow \texttt{INF}$
    \end{itemize}

    \item \textbf{Violaciones}:
    \begin{itemize}
        \item $g_k$: $\max(g_k(x), 0) \Rightarrow$ solo violaciones positivas
        \item $h_k$: $L^2$-norm del desvío
    \end{itemize}

    \item \textbf{Penalización catastrófica}:
    \begin{itemize}
        \item Activa para simulaciones físicamente inviables
    \end{itemize}
\end{itemize}

\subsubsection{Manejo de Datos Específico}
\begin{itemize}
    \item \textbf{Entradas}:
    \begin{itemize}
        \item $x \in {[\min_i, \max_i]}^n$
        \item $\lambda_p > 0$
    \end{itemize}

    \item \textbf{Intermedios}:
    \begin{itemize}
        \item $LE_{\text{calc}} \in \mathbb{R}$
        \item $\text{violaciones} \in \mathbb{R}^+$
    \end{itemize}

    \item \textbf{Salida}:
    \begin{itemize}
        \item $F_p(x) \in \mathbb{R}$
    \end{itemize}
\end{itemize}

\subsubsection{Salidas Principales}
\begin{itemize}
    \item \textbf{Fitness penalizado}: $F_p(x)$ ordenable para selección
\end{itemize}

\subsubsection{Interacciones Internas}
\begin{itemize}
    \item \textbf{REBOUND}: Provee $\{\text{pos}(t), \text{vel}(t)\}$
    \item \textbf{Módulo LE}: Calcula $f(x)$
    \item \textbf{Gestor de restricciones}: Evalúa $g_k(x)$, $h_j(x)$
\end{itemize}

\subsubsection{Diagrama del Proceso}
\begin{figure}[H]
    \centering
    \includegraphics[width=\textwidth]{img/Analisis/DiagramaProcesos/DiagramaProceso13_CalcularFitness.png}
    \caption{Diagrama de Proceso Interno 13: Calcular Fitness}%
    \label{fig:process_diagram13}
\end{figure}
