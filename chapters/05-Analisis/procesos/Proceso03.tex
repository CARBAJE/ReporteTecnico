\subsection{Proceso Interno 03: Generar Nueva Generación}

\subsubsection{Objetivo del Proceso}
El propósito principal de la actividad \textbf{``Generar Nueva Generación''} es crear una nueva población de $N$ individuos mediante operadores genéticos (Selección por Torneo, Cruzamiento SBX y Mutación Polinomial), garantizando que los parámetros permanezcan dentro de los rangos $[\min, \max]$. La evaluación de fitness se realiza posteriormente.

\subsubsection{Entradas Principales}
\begin{itemize}
    \item \textbf{Población actual}:
    \begin{itemize}
        \item $N$ individuos con parámetros y $F_p$ (fitness penalizado)
    \end{itemize}
    \item \textbf{ConfigurationData}:
    \begin{itemize}
        \item $N, P_c, P_m, \eta_c, \eta_m, k$
        \item Rangos $[\min_i, \max_i]$ por parámetro
    \end{itemize}
\end{itemize}

\subsubsection{Sub-pasos Secuenciales}
Este apartado es proporcionado para profundizar y describir de forma textual cada paso contenido dentro del diagrama del proceso descrito en la figura~\ref{fig:process_diagram03}
\subsubsection*{1. Inicializar Nueva Población}
\begin{itemize}
    \item Crear \texttt{new\_population} vacío (capacidad $N$)
\end{itemize}

\subsubsection*{2. Iniciar Bucle de Generación}
\begin{enumerate}[label= (\alph*)]
    \item \textbf{While} $|\texttt{new\_population}| < N$:
\end{enumerate}

\noindent\fbox{%
    \parbox{\textwidth}{%
    \textbf{Paso 2a: Selección de padres}
    \begin{itemize}
        \item \texttt{Parent1} = Torneo~($k$, $F_p$)
        \item \texttt{Parent2} = Torneo~($k$, $F_p$)
    \end{itemize}
    }%
}

\vspace{0.5em}
\noindent\fbox{%
    \parbox{\textwidth}{%
    \textbf{Paso 2b: Cruzamiento SBX}
    \begin{itemize}
        \item Generar $r_1 \sim U[0,1]$
        \item \textbf{If} $r_1 < P_c$:
            \begin{itemize}
                \item $\{\texttt{Offspring1}, \texttt{Offspring2}\} \leftarrow \text{SBX}(\texttt{Parent1}, \texttt{Parent2}, \eta_c)$
            \end{itemize}
        \item \textbf{Else}:
            \begin{itemize}
                \item Clonar padres
            \end{itemize}
    \end{itemize}
    }%
}

\vspace{0.5em}
\noindent\fbox{%
    \parbox{\textwidth}{%
    \textbf{Paso 2c: Mutación Polinomial}

    Para cada descendiente en $\{\texttt{Offspring1}, \texttt{Offspring2}\}$:

    Para cada parámetro $x_i$ del descendiente:
    \begin{itemize}
        \item Generar $r_2 \sim U[0,1]$
        \item \textbf{If} $r_2 < P_m$:
            \begin{itemize}
                \item $x_i \leftarrow \text{MutaciónPolinomial}(x_i, \eta_m, \text{min}_i, \text{max}_i)$
            \end{itemize}
    \end{itemize}
    }%
}

\vspace{0.5em}
\noindent\fbox{%
    \parbox{\textwidth}{%
    \textbf{Paso 2d: Ajuste de límites}

    Para cada parámetro $x_i$ en cada descendiente:
    \begin{itemize}
        \item $x_i \leftarrow \text{clip}(x_i, \text{min}_i, \text{max}_i)$
    \end{itemize}
    }%
}

\vspace{0.5em}
\noindent\fbox{%
    \parbox{\textwidth}{%
    \textbf{Paso 2e: Añadir descendientes}
    \begin{itemize}
        \item Agregar \texttt{Offspring1} a \texttt{new\_population}
        \item Si $|\texttt{new\_population}| < N$, agregar \texttt{Offspring2}
    \end{itemize}
    }%
}

\subsubsection*{3. Devolver Nueva Población}
\begin{itemize}
    \item Retornar \texttt{new\_population} con $N$ individuos
\end{itemize}

\subsubsection{Lógica Interna y Decisiones}
\begin{itemize}
    \item \textbf{Control del bucle}:
    \begin{itemize}
        \item Condición de término: $|\texttt{new\_population}| = N$
        \item Manejo de $N$ impar con adición condicional de \texttt{Offspring2}
    \end{itemize}

    \item \textbf{Operadores probabilísticos}:
    \begin{itemize}
        \item Cruzamiento: $P_c \in [0,1]$
        \item Mutación: $P_m \in [0,1]$ por parámetro
    \end{itemize}

    \item \textbf{Preservación de restricciones}:
    \begin{itemize}
        \item Ajuste post-operadores: $x_i \in [\text{min}_i, \text{max}_i]$
    \end{itemize}
\end{itemize}

\subsubsection{Manejo de Datos Específico}
\begin{itemize}
    \item \textbf{Entradas}:
    \begin{itemize}
        \item Población actual: Lista de estructuras $\{\text{parámetros}, F_p\}$
        \item Parámetros operadores: $\{P_c, P_m, \eta_c, \eta_m, k\}$
    \end{itemize}

    \item \textbf{Intermedios}:
    \begin{itemize}
        \item Padres seleccionados
        \item Descendientes pre-ajuste
    \end{itemize}

    \item \textbf{Salida}:
    \begin{itemize}
        \item \texttt{new\_population}: Lista de $N$ individuos no evaluados
    \end{itemize}
\end{itemize}

\subsubsection{Salidas Principales}
\begin{itemize}
    \item \textbf{Nueva población}:
    \begin{itemize}
        \item $N$ individuos con parámetros en rangos válidos
        \item Listos para evaluación de fitness
    \end{itemize}
\end{itemize}

\subsubsection{Interacciones Internas}
\begin{itemize}
    \item \textbf{Con población actual}: Lectura de $F_p$ para torneos
    \item \textbf{Con módulos genéticos}:
    \begin{itemize}
        \item Torneo~($k$), SBX~($\eta_c$), MutaciónPolinomial~($\eta_m$)
    \end{itemize}
    \item \textbf{Con gestor de restricciones}: Aplicación de $\text{clip}()$
\end{itemize}
\newpage
\subsubsection{Diagrama del Proceso}
\begin{figure}[H]
    \centering
    \adjustbox{max width=\textwidth, max height=0.9\textheight}{%
        \includegraphics{img/Analisis/DiagramaProcesos/DiagramaProceso03_GenerarNuevaGeneración.png}
    }
        \caption{Diagrama de Proceso Interno 03: Generar Nueva Generación}%
    \label{fig:process_diagram03}
\end{figure}
\newpage