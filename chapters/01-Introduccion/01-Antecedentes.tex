\section{Antecedentes}
El desarrollo de simulaciones de \textit{n}-cuerpos ha sido ampliamente investigado con el objetivo de mejorar su capacidad para modelar fenómenos complejos en entornos como la colisión de galaxias y la evolución de sistemas planetarios. Sin embargo, una limitación crítica es la incapacidad de ajustar los parámetros de los cuerpos durante la ejecución de la simulación, lo que afecta la representación precisa de eventos masivos como colisiones en las simulaciones disponibles.

El problema de los \textit{n}-cuerpos\footnote{El problema de \textit{n}-cuerpos se refiere al estudio del movimiento e interacción gravitacional de varios cuerpos bajo sus influencias mutuas. En la sección~\ref{sec:n-body_problem} se profundiza a fondo el termino.} ha sido uno de los desafíos más complejos en la mecánica celeste, desde los trabajos pioneros de Newton en el siglo XVII.\ Si bien el problema de los dos cuerpos tiene una solución analítica exacta bajo ciertas condiciones, cuando se introducen factores adicionales, se requieren métodos numéricos avanzados para mejorar la precisión de las simulaciones.

Avances como el método multipolar rápido (FMM, por sus siglas en inglés, \textit{Fast Multipole Method})\footnote{El FMM es un algoritmo que optimiza el cálculo de interacciones entre partículas distantes en sistemas físicos, como los gravitacionales, agrupando partículas y aproximando su influencia colectiva, lo que reduce la complejidad computacional de \(O(n^2)\) a \(O(n)\) o \(O(n \log n)\).} y el algoritmo de Barnes-Hut\footnote{El algoritmo de Barnes-Hut es una técnica de simulación de \textit{n}-cuerpos que aproxima las fuerzas gravitacionales al agrupar cuerpos distantes en una misma región del espacio, representándolos como una única masa, lo que reduce la complejidad computacional de \(O(n^2)\) a \(O(n \log n)\).} han optimizado las simulaciones de \textit{n}-cuerpos al reducir la complejidad computacional en el cálculo de las interacciones gravitacionales. Sin embargo, la falta de capacidad para ajustar dinámicamente las propiedades de los cuerpos durante la ejecución sigue siendo una limitación en la representación realista de eventos astronómicos.

Estudios recientes han introducido herramientas y enfoques que podrían mejorar las simulaciones de \textit{n}-cuerpos, como el uso de árboles cuádruples y octales para la representación geométrica y la aplicación de inteligencia artificial en las simulaciones moleculares que podrían inspirar nuevas técnicas en simulaciones celestes. Además, se ha demostrado la eficacia de integradores simplécticos\footnote{Los integradores simplécticos son métodos numéricos utilizados para resolver ecuaciones diferenciales en sistemas dinámicos conservando las propiedades geométricas del sistema, como la conservación de la energía a largo plazo, lo que los hace especialmente adecuados para simular interacciones físicas, como las gravitacionales, en sistemas de \textit{n}-cuerpos.} para garantizar la estabilidad de las simulaciones durante colisiones y la paralelización para mejorar la eficiencia en simulaciones masivas.