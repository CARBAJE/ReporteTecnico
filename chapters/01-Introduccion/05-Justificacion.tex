\section{Justificaión}
Este proyecto es relevante porque introduce un enfoque innovador en la simulación de sistemas gravitacionales al permitir la modificación dinámica de parámetros, superando las limitaciones de los modelos tradicionales. La elección de algoritmos avanzados, como el método multipolar rápido (FMM) y el algoritmo de Barnes-Hut, garantiza un equilibrio entre precisión y eficiencia computacional, lo que permite su aplicación en escenarios más complejos sin un costo computacional excesivo. Además, la implementación de técnicas de optimización bioinspiradas ofrece un método adaptable para ajustar el sistema en tiempo real, mejorando la fidelidad de las simulaciones.

El uso de estas tecnologías no solo optimiza el rendimiento de la simulación, sino que también sienta las bases para su posible escalabilidad a problemas de mayor complejidad, como la simulación de múltiples cuerpos. Aunque el presente trabajo se centra en el problema de dos cuerpos, su enfoque y metodología podrían aplicarse en otros ámbitos, como el modelado de interacciones gravitacionales en videojuegos y simulaciones interactivas.

Los principales beneficiarios de este proyecto serán investigadores y académicos en los campos de la astronomía, astrofísica y mecánica celeste, quienes podrán utilizar el modelo para mejorar el análisis y la comprensión de interacciones gravitacionales. Además, su posible aplicación en videojuegos y simulaciones interactivas podría impactar significativamente la industria del entretenimiento y la educación, proporcionando herramientas más flexibles y precisas para la enseñanza y la creación de simulaciones realistas.

Asimismo, la industria aeroespacial y los organismos encargados de planificar maniobras espaciales podrían beneficiarse del modelo, ya que permitiría simular eventos dinámicos, como colisiones orbitales o ajustes en la trayectoria de satélites. Al ofrecer la posibilidad de modificar dinámicamente los parámetros de los cuerpos en simulación, esta herramienta facilitaría la planificación y optimización de maniobras espaciales, mejorando la toma de decisiones en distintos tipos de misiones.