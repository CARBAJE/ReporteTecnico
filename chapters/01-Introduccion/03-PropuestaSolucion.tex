\section{Propuesta de solución}

La solución propuesta se basa en el desarrollo de un modelo que permite la modificación dinámica de parámetros durante la simulación, lo que representa una mejora significativa con respecto a los simuladores tradicionales, que requieren reconfiguraciones antes de cada ejecución. Este modelo integrará técnicas avanzadas para optimizar el cálculo de interacciones gravitacionales y garantizar la estabilidad de la simulación.

Para lograrlo, se emplearán distintos métodos de cálculo sobre fuerzas gravitacionales y cálculos de colisión. Respecto a los cálculos de fuerzas gravitacionales, uno de los elementos plausibles a integrar es el método multipolar rápido (FMM) y el algoritmo de Barnes-Hut, ambos ampliamente utilizados para reducir la complejidad computacional en simulaciones de \textit{n}-cuerpos sin comprometer la precisión.\ Estas técnicas permiten agrupar cuerpos en estructuras jerárquicas, disminuyendo el número de cálculos directos y mejorando la eficiencia del sistema.

Además, la implementación de algoritmos bioinspirados, como la optimización por enjambre de partículas (PSO) y algoritmos genéticos (AGs), permitirá el ajuste dinámico de los parámetros de los cuerpos celestes, garantizando un comportamiento estable incluso en escenarios de colisión o interacciones complejas. Estas estrategias facilitarán la identificación y ajuste de valores óptimos sin necesidad de intervención manual, lo que hará que la simulación sea más adaptable a eventos inesperados.

El modelo se diseñará para ser escalable, permitiendo la modificación de los parámetros, conjunto de parámetros que solo involucra a la masa de los cuerpos dada las limitaciones de tiempo para el trabajo terminal, de hasta dos cuerpos celestes, con la posibilidad de extenderse a más cuerpos en implementaciones futuras. Su integración con entornos virtuales como Unreal Engine o motores de simulación especializados permitirá su aplicación en campos como la educación, los videojuegos y la industria aeroespacial; dicha implementación se deja para mejoras futuras al proyecto.

Para garantizar un rendimiento eficiente, el modelo se desarrollará optimizado para ejecutarse en hardware de gama media, como procesadores multinúcleo con al menos 16 GB de RAM.\ Esto asegurará que la solución sea accesible sin requerir infraestructuras de alto rendimiento, ampliando su potencial uso en distintos ámbitos.

Esta combinación de técnicas avanzadas y adaptabilidad en tiempo real establece una base innovadora para el desarrollo de simulaciones gravitacionales más dinámicas, eficientes y versátiles, superando las limitaciones de los modelos preconfigurados actuales.