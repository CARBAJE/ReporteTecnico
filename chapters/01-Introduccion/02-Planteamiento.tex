\section{Planteamiento del problema}
Las simulaciones de sistemas de \textit{n}-cuerpos celestes han sido una herramienta fundamental en la astronomía y la mecánica celeste para estudiar la evolución de sistemas planetarios, la dinámica estelar y la interacción gravitacional a gran escala. Sin embargo, una limitación crítica en los modelos actuales es la imposibilidad de modificar dinámicamente los parámetros de los cuerpos durante la ejecución de la simulación. En la mayoría de los simuladores, estos parámetros, como la masa, la energía del sistema, la posición y el tiempo de existencia de los cuerpos se definen antes de la ejecución, impidiendo la representación precisa de eventos atípicos o transitorios, como colisiones de galaxias, acreción de materia en discos protoplanetarios o cambios de masa en estrellas variables.

Uno de los problemas más importantes derivados de esta limitación es la incapacidad de ajustar la masa de los cuerpos celestes en tiempo real. En eventos como fusiones de agujeros negros o estrellas en procesos de acreción, la masa de los cuerpos cambia significativamente a lo largo del tiempo, alterando la evolución del sistema. Sin la posibilidad de modificar este parámetro de manera dinámica, los modelos actuales ofrecen solo aproximaciones estáticas que no reflejan con precisión la naturaleza de estos fenómenos.

Además, el rendimiento computacional también representa un desafío. La simulación de \textit{n}-cuerpos es un problema computacionalmente costoso, ya que la cantidad de interacciones a calcular crece cuadráticamente con el número de cuerpos (O(n²)), lo que hace que las simulaciones a gran escala sean prohibitivas en términos de tiempo de ejecución y recursos computacionales. Métodos como el algoritmo de Barnes-Hut y el método multipolar rápido han sido desarrollados para mitigar este problema reduciendo la cantidad de cálculos directos, pero no abordan la falta de flexibilidad en la modificación de parámetros en tiempo real.

En este contexto, la incapacidad de modificar parámetros dinámicamente en simulaciones de \textit{n}-cuerpos afecta no solo la precisión de la representación de fenómenos astronómicos, sino también la capacidad de realizar simulaciones interactivas en tiempo real, algo que podría tener aplicaciones en la enseñanza, la exploración espacial y la industria del entretenimiento. La ausencia de modelos que permitan esta flexibilidad limita el alcance de las simulaciones actuales y plantea la necesidad de desarrollar enfoques más adaptativos y eficientes.